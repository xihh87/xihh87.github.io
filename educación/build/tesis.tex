% Options for packages loaded elsewhere
\PassOptionsToPackage{unicode}{hyperref}
\PassOptionsToPackage{hyphens}{url}
%
\documentclass[
  12,
]{scrartcl}
\usepackage{amsmath,amssymb}
\usepackage{lmodern}
\usepackage{iftex}
\ifPDFTeX
  \usepackage[T1]{fontenc}
  \usepackage[utf8]{inputenc}
  \usepackage{textcomp} % provide euro and other symbols
\else % if luatex or xetex
  \usepackage{unicode-math}
  \defaultfontfeatures{Scale=MatchLowercase}
  \defaultfontfeatures[\rmfamily]{Ligatures=TeX,Scale=1}
  \setmainfont[]{DejaVu Sans}
\fi
% Use upquote if available, for straight quotes in verbatim environments
\IfFileExists{upquote.sty}{\usepackage{upquote}}{}
\IfFileExists{microtype.sty}{% use microtype if available
  \usepackage[]{microtype}
  \UseMicrotypeSet[protrusion]{basicmath} % disable protrusion for tt fonts
}{}
\makeatletter
\@ifundefined{KOMAClassName}{% if non-KOMA class
  \IfFileExists{parskip.sty}{%
    \usepackage{parskip}
  }{% else
    \setlength{\parindent}{0pt}
    \setlength{\parskip}{6pt plus 2pt minus 1pt}}
}{% if KOMA class
  \KOMAoptions{parskip=half}}
\makeatother
\usepackage{xcolor}
\IfFileExists{xurl.sty}{\usepackage{xurl}}{} % add URL line breaks if available
\IfFileExists{bookmark.sty}{\usepackage{bookmark}}{\usepackage{hyperref}}
\hypersetup{
  pdftitle={Aplicación del paradigma constructivista a la enseñanza de la ciberseguridad en México usando actividades dirigidas e integración continua---Primer borrador},
  pdfauthor={Joshua Ismael Haase Hernández},
  pdflang={es-MX},
  hidelinks,
  pdfcreator={LaTeX via pandoc}}
\urlstyle{same} % disable monospaced font for URLs
% Use graphics on front page
\usepackage{graphicx}
% Make links footnotes instead of hotlinks:
\DeclareRobustCommand{\href}[2]{#2\footnote{\url{#1}}}
\usepackage[normalem]{ulem}
% Avoid problems with \sout in headers with hyperref
\pdfstringdefDisableCommands{\renewcommand{\sout}{}}
\setlength{\emergencystretch}{3em} % prevent overfull lines
\providecommand{\tightlist}{%
  \setlength{\itemsep}{0pt}\setlength{\parskip}{0pt}}
\setcounter{secnumdepth}{-\maxdimen} % remove section numbering
\newlength{\cslhangindent}
\setlength{\cslhangindent}{1.5em}
\newlength{\csllabelwidth}
\setlength{\csllabelwidth}{3em}
\newlength{\cslentryspacingunit} % times entry-spacing
\setlength{\cslentryspacingunit}{\parskip}
\newenvironment{CSLReferences}[2] % #1 hanging-ident, #2 entry spacing
 {% don't indent paragraphs
  \setlength{\parindent}{0pt}
  % turn on hanging indent if param 1 is 1
  \ifodd #1
  \let\oldpar\par
  \def\par{\hangindent=\cslhangindent\oldpar}
  \fi
  % set entry spacing
  \setlength{\parskip}{#2\cslentryspacingunit}
 }%
 {}
\usepackage{calc}
\newcommand{\CSLBlock}[1]{#1\hfill\break}
\newcommand{\CSLLeftMargin}[1]{\parbox[t]{\csllabelwidth}{#1}}
\newcommand{\CSLRightInline}[1]{\parbox[t]{\linewidth - \csllabelwidth}{#1}\break}
\newcommand{\CSLIndent}[1]{\hspace{\cslhangindent}#1}
\ifLuaTeX
\usepackage[bidi=basic]{babel}
\else
\usepackage[bidi=default]{babel}
\fi
\babelprovide[main,import]{spanish}
% get rid of language-specific shorthands (see #6817):
\let\LanguageShortHands\languageshorthands
\def\languageshorthands#1{}
\ifLuaTeX
  \usepackage{selnolig}  % disable illegal ligatures
\fi

\title{Aplicación del paradigma constructivista a la enseñanza de la
ciberseguridad en México usando actividades dirigidas e integración
continua---Primer borrador}
\author{Joshua Ismael Haase Hernández}
\date{2021-10-19}

\begin{document}
\begin{titlepage}
 \begin{addmargin}[-1.5cm]{-1cm}
	\centering
        \includegraphics[width=13cm]{uco.png}\par
        \vspace{2\baselineskip}
        {\LARGE \textsc{Doctorado en Educación}\par}
        \vspace{2\baselineskip}
        {\Huge \textsc{Aplicación del paradigma constructivista a la
enseñanza de la ciberseguridad en México usando actividades dirigidas e
integración continua---Primer borrador}\par}
        por\par
                {\Large\textsc{Joshua Ismael Haase Hernández}\par}
                \vspace{2\baselineskip}
        \vfill
        {\Large Ciudad de México, 2021-10-19\par}
 \end{addmargin}
\end{titlepage}

\renewcommand*\contentsname{Contenido}
{
\setcounter{tocdepth}{3}
\tableofcontents
}
\newpage

\hypertarget{descripciuxf3n-del-proyecto}{%
\section{Descripción del proyecto}\label{descripciuxf3n-del-proyecto}}

\begin{quote}
``Me lo contaron y lo olvidé, lo vi y lo entendí, lo hice y lo
aprendí.''

Confucio (jeronicalafell, 2017)
\end{quote}

Desarrollar un conjunto de juegos para guiar a los estudiantes a
entender y aplicar los conceptos de ciberseguridad, desarrollar
pensamiento sistémico y sistemático, y reconocer áreas de oportunidad en
términos de la seguridad informática.

A diferencia de utilizar una aproximación meramente teórica este
proyecto ofrecería una ruta optimizada para el aprendizaje de una
asignatura técnicamente retadora y generar experiencia práctica en el
área.

\hypertarget{planteamiento-del-problema}{%
\section{Planteamiento del problema}\label{planteamiento-del-problema}}

\begin{quote}
\begin{enumerate}
\def\labelenumi{\arabic{enumi}.}
\item
  Reunir los hechos en relación con el problema ¿Qué está pasando?

  En el contexto local:
\end{enumerate}
\end{quote}

Los proyectos informáticos comienzan con el objetivo de resolver un
problema y los programadores que no han adquirido la disciplina de
proteger cada entrada de información en los programas, introducen
vulnerabilidades buscando la forma más simple de resolver el problema.

Dado que la mayoría de los programadores no reciben instrucción formal
(\textbf{bob\_c\_martin\_youtube?}) la mayoría de los programadores no
desarrollan esa disciplina y muchos de los materiales en internet
incluso recomiendan agregar vulnerabilidades en el desarrollo para
simplificar, sin indicar los problemas que ocasionan.

En la Universidad Iberoamericana se busca enseñar esta disciplina a los
estudiantes como una de las ventajas competitivas del programa ITCT.

\begin{quote}
En el contexto global:
\end{quote}

\begin{itemize}
\item
  El número de computadoras y su influencia en nuestra vida está en
  constante creciendo.
\item
  Hay más demanda que oferta en el mercado laboral para personas
  capacitadas en programación, seguridad informática, inteligencia
  artificial.
\item
  El uso de inteligencia artificial para automatizar tareas puede
  destruir empleos mucho más rápido de lo que generamos empleos nuevos y
  se espera que los empleos nuevos que se generen requieran un alto
  nivel técnico para desarrollarse.

  Esto pasará principalmente en los mandos medios porque es dónde tendrá
  más valor en cuanto al balance costo/beneficio para las empresas.
\item
  Muchas habilidades creativas mantendrán su valor si no pueden
  automatizarse, aunque la parte técnica de esas carreras será apoyada
  por sistemas automatizados.
\item
  Se espera que los sistemas automatizados se vuelvan ubicuos y que la
  atención personalizada se vuelva producto de lujo. Por esta razón, la
  adquisición de habilidades blandas (comunicación, empatía, etcétera)
  tendrá alto valor en el mercado laboral.
\end{itemize}

\begin{quote}
\begin{enumerate}
\def\labelenumi{\arabic{enumi}.}
\setcounter{enumi}{1}
\tightlist
\item
  Determinar la importancia de los hechos.
\end{enumerate}
\end{quote}

El uso de las nuevas tecnologías tiene importancia en el ámbito
económico y social de nuestras vidas por su impacto en los mercados
laborales.

El uso de las nuevas tecnologías tiene impacto en muchas industrias
porque cambia el tamaño de la escala óptima para desarrollar varias
actividades. Por ejemplo, actualmente una persona puede tener impacto
mediático a nivel nacional o global, sin todo el aparato de una
institución.

Para las escuelas, el uso de tecnologías de la información para difundir
el conocimiento y la disponibilidad de la información en línea implica
un cambio de su estatus como el lugar dónde se difunde el conocimiento.

-\textgreater{} Esto probablemente se mitiga por el hecho de que muchas
personas no son autodidactas.

-\textgreater{} También puede ser que la necesidad de evaluación mitigue
este problema.

\begin{quote}
\begin{enumerate}
\def\labelenumi{\arabic{enumi}.}
\setcounter{enumi}{2}
\tightlist
\item
  Identificar las posibles relaciones entre los hechos que puedan
  indicar la causa de la dificultad.
\end{enumerate}
\end{quote}

La ubicuidad de las computadoras y el desarrollo de las tecnologías
(p.e. paralelización, GPU) permite que sea posible automatizar tareas
que se supone requieren criterio o inteligencia.

Esto está provocando:

\begin{itemize}
\tightlist
\item
  Automatización de tareas de alto valor agregado
\item
  Demanda de personal capacitado en habilidades técnicas
\item
  Destrucción o transformación de los empleos que se automatizan
\end{itemize}

Esas situaciones implican un cambio en el mercado laboral, en la
cantidad y calidad de trabajo que pueden entregar las personas que sepan
usar o tengan acceso a estas tecnologías, en la capacidad técnica que
van a requerir los empleos para hacer funcionar la tecnología y en la
demanda de personal especializado para implementar sistemas de
aprendizaje de máquina.

Todas estas tendencias están creando demanda de personal que pueda
utilizar estas nuevas tecnologías y se necesita un medio efectivo para
ayudar a las personas que podrían cubrir estas demandas a capacitarse.

Recuerdo haber leído en un informe o una referencia a un informe de
Gartner que la demanda de personal especializado está creciendo más
rápido de lo que las escuelas están formando al personal que podría
atenderla.

\begin{quote}
\begin{enumerate}
\def\labelenumi{\arabic{enumi}.}
\setcounter{enumi}{3}
\tightlist
\item
  Proponer explicaciones para conocer la causa de la dificultad y
  determinar su importancia en el problema.
\end{enumerate}
\end{quote}

Parece que nunca antes en la historia de la humanidad ha habido una
demanda tan grande de las habilidades de abstracción y pensamiento
lógico-matemático. Los métodos que se han desarrollado para ayudar a las
personas a desarrollar estas habilidades están basados en el diseño de
las fábricas, impulsados por el Taylorismo, la necesidad de entrenar
personal para la producción industrial y la producción en serie.

Entre las teorías educativas, las que parecen atender el problema de
cómo se generan las abstracciones son el cognoscitivismo y el
constructivismo.

Ambos tienen una aproximación práctica para acelerar el proceso:

\begin{itemize}
\tightlist
\item
  Organizando el conocimiento en función de los conceptos que deben
  desarrollarse.
\item
  Generando retos en el horizonte proximal de conocimiento sin mostrar
  todo el resultado, sino dejando espacio para el descubrimiento.
\item
  Usando actividades que den la oportunidad a los estudiantes para
  desarrollar la abstracción.
\end{itemize}

\begin{quote}
\begin{enumerate}
\def\labelenumi{\arabic{enumi}.}
\setcounter{enumi}{4}
\tightlist
\item
  Encontrar entre las explicaciones aquellas que permitan adquirir una
  visión amplia en la solución del problema.
\end{enumerate}
\end{quote}

La teoría constructivista da pistas acerca de cómo se desarrollan las
abstracciones necesarias para aprender el uso de las nuevas tecnologías.

El poder de las tecnologías de información como un medio para la
transmisión de conocimiento, con posibilidades de ejecutar acciones o
interactuar con las personas de acuerdo a patrones definidos, las
vuelven una herramienta idónea para abordar el problema.

\begin{quote}
\begin{enumerate}
\def\labelenumi{\arabic{enumi}.}
\setcounter{enumi}{5}
\tightlist
\item
  Hallar relaciones entre los hechos y las explicaciones.
\end{enumerate}
\end{quote}

El problema está causado por el desarrollo tecnológico y se están
explorando varios modelos para atacar el problema.

\begin{quote}
\begin{enumerate}
\def\labelenumi{\arabic{enumi}.}
\setcounter{enumi}{6}
\tightlist
\item
  Analizar los supuestos en los que se apoyan los elementos
  identificados.
\end{enumerate}
\end{quote}

Las tecnologías de la información están permitiendo a muchas personas
ser autodidactas y montar su propio negocio.

En algunas regiones del mundo, los costos de las universidades privadas
están teniendo un retorno de inversión negativo para muchas personas y
se está proponiendo que no son necesarias porque las tecnologías de la
información permiten formarse de manera autodidacta con una barrera de
entrada menor a la que antes existía.

Aún si las tecnologías de la información no reemplazan el rol de las
universidades, las estamos desaprovechando si no potenciamos la labor
docente al hacer uso de ellas.

Las tecnologías de la información pueden ser mejores que los libros,
porque tienen un alcance mayor, mayor potencial y su mayor ventaja es
que pueden diseñarse sistemas que ofrezcan retroalimentación en tiempo
real a los estudiantes.

Además de ser el medio por el cuál se desarrollan las mismas técnicas de
aprendizaje de máquina en las que queremos entrenar a las personas.
Estas herramientas pueden ser tanto el medio de exposición y enseñanza
para estas nuevas tecnologías.

\hypertarget{justificaciuxf3n}{%
\section{Justificación}\label{justificaciuxf3n}}

Acerca de la importancia de mejorar el proceso educativo integrando las
tecnologías de la información tomaré prestadas las palabras de Eben
Moglen en su conferencia «Antes y Después de la Propiedad Intelectual»:

\begin{quote}
"¿Cuántos Shakespeares pudo haber que no aprendieron a leer? ¿Cuantos
Einstein no aprendieron física? Es obvio que la mayoría, presumiblemente
bastantes.

\emph{Hay más humanos en el planeta que nunca antes, así que no importa
cuantos Shakespeares o Einsteins se hayan perdido antes, vamos a
desperdiciarlos otra vez.} Pero no tenemos que hacerlo.

A diferencia de todas las sociedades anteriores que estaban casi
obligadas por circunstancias materiales a desperdiciarlos {[}\ldots{]}
Ahora el único impedimento son las leyes que prohíben compartir, y sin
ellas la ignorancia sería prevenible."

(s.~f.), resaltado añadido.
\end{quote}

Las tecnologías de la información nos permiten alcanzar más personas y
hacer las cosas a gran escala por su capacidad de ejecutar procesos de
manera reproducible, automática, constante.

Las nuevas herramientas permiten además:

\begin{itemize}
\tightlist
\item
  Realizar actividades sin conocer cómo se hace,
\item
  Hacer trabajo de calidad con muchos menos recursos (tiempo, esfuerzo),
\item
  Multiplicar la capacidad personal.
\end{itemize}

El costo que pagamos por ello, es la necesidad de mayor capacitación y
entrenamiento para desarrollar capacidades de abstracción avanzadas, que
los modelos tradicionales de enseñanza basados en el conductismo no
están preparados para entrenar.

La habilidad de trabajar con las computadoras se puede entender como una
nueva etapa de alfabetización que divide a las personas más exitosas en
nuestra sociedad (\emph{dbp.io :: Programming as Literature}, s.~f.).

No es la única habilidad ni necesariamente la más importante, pero sí
representa una diferencia cualitativa en la calidad de vida que pueden
tener las personas.

\begin{enumerate}
\def\labelenumi{\arabic{enumi}.}
\tightlist
\item
  ¿Cuáles son los beneficios que este trabajo proporcionará?
\end{enumerate}

Para atender la creciente necesidad de profesionistas en el mercado, y
mejorar la eficacia del entrenamiento en las materias de programación,
este trabajo propone el uso de un sistema automático de enseñanza y
evaluación para las materias de programación que utiliza las mejores
prácticas de la industria para guiar a los estudiantes a través de su
aprendizaje.

\begin{enumerate}
\def\labelenumi{\arabic{enumi}.}
\setcounter{enumi}{1}
\tightlist
\item
  ¿Quiénes serán los beneficiarios y de qué modo?
\end{enumerate}

El sistema propuesto presenta los siguientes beneficios sobre los
métodos tradicionales de enseñanza:

\begin{itemize}
\item
  Ofrece retroalimentación inmediata, continua y consistente \emph{a los
  estudiantes} durante su proceso de aprendizaje.

  Esto podría ayudar a desarrollar buenos hábitos, que se han
  identificado como el principal factor que afecta el desempeño de los
  programadores (McConnell, 2004).

  Verificar esta hipótesis está contemplado dentro del alcance de este
  trabajo.
\item
  Introduce \emph{a los estudiantes} en la dinámica de desarrollo
  profesional de software desde su formación. Esto podría representar
  una ventaja competitiva para los estudiantes que al momento de buscar
  trabajo pueden presentar su experiencia utilizando sistemas de
  integración continua.

  Este rubro beneficia también \emph{a las empresas} porque reduce el
  tiempo de adaptación a las buenas prácticas en la industria.
\item
  Simplifica la labor \emph{docente} al descargar la responsabilidad de
  las actividades de evaluación que pueden evaluarse de manera
  automática.

  No es evidente que todos los atributos deseables en un buen código
  pueden evaluarse de manera automática. Muchos de los criterios
  requieren habilidades de comprensión de contexto que no tienen las
  computadoras; salvo que se puedan implementar como algoritmos de
  aprendizaje de máquina. Esta posibilidad está fuera del alcance de
  este trabajo, pero podría incluirse en trabajos posteriores.
\item
  Funciona como un estándar externo para \emph{la universidad} en la
  medición de la calidad del trabajo docente.

  Este sistema podría ofrecer parámetros objetivos para evaluar si los
  estudiantes desarrollan buenos hábitos en programación.
\end{itemize}

\begin{enumerate}
\def\labelenumi{\arabic{enumi}.}
\setcounter{enumi}{2}
\tightlist
\item
  ¿Qué aporta de nuevo esta investigación?
\end{enumerate}

Un sistema de integración continua es un conjunto de herramientas y
prácticas que se utiliza en la industria de desarrollo de software para:

\begin{itemize}
\tightlist
\item
  mantener la calidad del código,
\item
  agilizar el tiempo para entregar valor en forma de productos,
\item
  reducir el tiempo que toma detectar problemas y corregirlos.
\end{itemize}

Estos sistemas tienen beneficios bien probados en los ámbitos de la
industria (McConnell, 2004) e investigación científica (Krafczyk et~al.,
2019). Sin embargo, no suele incluirse en los currículos universitarios
(Bowyer \& Hughes, 2006).

Este trabajo se apropia de ese sistema utilizado en las empresas más
ágiles de la industria de desarrollo de software y lo enfoca para
evaluar:

\begin{itemize}
\tightlist
\item
  el estilo de código
\item
  buenas prácticas en desarrollo
\end{itemize}

Además sienta las bases para utilizar este sistema de integración
continua como instrumento pedagógico mediante el paradigma
constructivista.

Ha habido intentos de integrar los sistemas de integración continua como
temática en el currículo educativo (Eddy et~al., 2017), o como
herramienta para mejorar la calidad del código en cursos universitarios
(Embury \& Page, 2019; Zvacek et~al., 2014). Sin embargo, no plantean el
uso de estas herramientas dentro de un marco pedagógico que apoye el
aprendizaje de los fundamentos de programación y verifique buenas
prácticas en el estilo de código.

Existen otras aproximaciones para intentar calificar de manera
automática y motivar a los estudiantes por medio de competencia (s.~f.).
Parece que la competencia también puede tener un efecto negativo en el
aprendizaje, porque divide la auto-percepción de la capacidad de los
estudiantes en buena y mala y en algunos casos disminuye la motivación
de los estudiantes por lo que debe implementarse con cuidado (Fischer
et~al., 2021).

Debido a limitaciones de tiempo, en este trabajo sólo se evaluará el
efecto de la retroalimentación continua con respecto de los criterios de
evaluación, pero se pretende extender este trabajo hasta implementar
ejercicios consistentes con el paradigma constructivista y el
aprendizaje por descubrimiento, en una secuencia que apoye el desarrollo
de los conceptos abstractos necesarios para programar y el desarrollo de
hábitos efectivos.

Cuando esa etapa llegue, se hará uso de plataformas ya existentes cuya
licencia permita el uso para fines educativos (Zinovieva et~al., 2021).

\begin{enumerate}
\def\labelenumi{\arabic{enumi}.}
\setcounter{enumi}{3}
\tightlist
\item
  ¿Qué es lo que se prevé cambiar con la investigación?
\end{enumerate}

Las universidades no han cambiado mucho desde el diseño de las
industrias con base a la administración científica (Taylor et~al., 2003)
y la producción en serie (González Martínez, 2003).

Las tecnologías de la información están cambiando las reglas del juego
(Levy \& Murnane, 2005) de tal forma que las universidades necesitarán
transformarse para cubrir la demanda de personal técnico especializado o
perderán su lugar en el mercado (s.~f.).

Es importante valernos de las tecnologías para resolver estos problemas
porque por un lado hay una creciente demanda de habilidades cognitivas
complejas (s.~f.) y por otro tenemos posibilidades tecnológicas que no
hubo nunca antes.

Este trabajo pretende cambiar la forma en que se enseña programación, y
en particular:

\begin{itemize}
\item
  Mejorar en la Universidad que trabajo la calidad e inmediatez de la
  retroalimentación que reciben los estudiantes.
\item
  Generar un sistema estandarizado y automático para la evaluación.
\end{itemize}

\begin{enumerate}
\def\labelenumi{\arabic{enumi}.}
\setcounter{enumi}{4}
\tightlist
\item
  ¿Cuál es su utilidad?
\end{enumerate}

Si el método es efectivo, mejorará la efectividad del aprendizaje en
estas materias que por su complejidad tienen un alto índice de
deserción.

\begin{enumerate}
\def\labelenumi{\arabic{enumi}.}
\setcounter{enumi}{5}
\tightlist
\item
  ¿Ayudará a resolver algún problema o gama de problemas prácticos?
\end{enumerate}

Se espera que reduzca el tiempo que los docentes dedican a la evaluación
al tiempo que mejor la cantidad, calidad y consistencia de la
retroalimentación que reciben los estudiantes.

\begin{enumerate}
\def\labelenumi{\arabic{enumi}.}
\setcounter{enumi}{6}
\tightlist
\item
  ¿Por qué es significativo este problema de investigación?
\end{enumerate}

Encontrar maneras efectivas para enseñar las disciplinas complejas y
abstractas que requiere una sociedad que cada vez depende más de las
computadoras puede tener impacto en beneficio de la sociedad:

\begin{itemize}
\item
  Entrenando a la población en un ámbito que tiene gran potencial para
  fortalecer el crecimiento económico y apoyar en la resolución de
  problemas relevantes para la sociedad que requieran automatización o
  complejidad de cómputo.
\item
  Preparando a la población que tiene las herramientas para resolver los
  problemas complejos del futuro.

  Los problemas científicos actuales ya son tan complejos que requieren
  el uso de computadoras para avanzar el conocimiento. Lo mismo para los
  problemas sociales con causas complejas como el calentamiento global.
\item
  Ofreciendo oportunidades para mejorar la calidad de vida de los
  estudiantes.

  La entrada en el mercado de las tecnologías de aprendizaje de máquina
  ya están cambiando el mercado, destruyendo ciertas profesiones y
  generando otras. Se espera que esta tendencia se acelere.
\item
  Formar estudiantes preparados para este futuro va a tener incidencia
  en la calidad de vida de los estudiantes que se están formando ahora.
\end{itemize}

\begin{enumerate}
\def\labelenumi{\arabic{enumi}.}
\setcounter{enumi}{7}
\tightlist
\item
  ¿Permitirá llenar algún hueco de conocimiento?
\end{enumerate}

Hernández (2012) argumenta que la evaluación continua sólo mejora el
desempeño estudiantil si se diseña la retroalimentación hacia el
aprendizaje y se deja claro qué tienen que hacer los estudiantes con la
retroalimentación recibida.

Hay varios ejemplos que muestran que la evaluación continua puede
mejorar el desempeño estudiantil en campos como ingeniería mecánica,
ingeniería química y medicina (Aftab \& Tariq, 2018; Christoforou \&
Yigit, 2008; Sanz-Pérez, 2019)

\begin{enumerate}
\def\labelenumi{\arabic{enumi}.}
\setcounter{enumi}{8}
\tightlist
\item
  ¿Se podrán generalizar los resultados a principios más amplios?
\end{enumerate}

Estos resultados podrían ser generalizables a situaciones educativas
dónde:

\begin{itemize}
\tightlist
\item
  Existan problemas bien definidos, susceptibles de calificarse
  automáticamente.
\item
  Requieran desarrollar habilidades de abstracción para resolverse.
\item
  Se pueda aprovechar un sistema de retroalimentación.
\item
  Los modos de fallo sean conocidos y enumerables.
\end{itemize}

\begin{enumerate}
\def\labelenumi{\arabic{enumi}.}
\setcounter{enumi}{9}
\tightlist
\item
  ¿Puede servir para comentar, desarrollar o apoyar una teoría?
\end{enumerate}

En esta etapa de evaluación, no se desarrolla una teoría nueva, pero se
sientan las bases para en el futuro evaluar una secuencia didáctica
basada en el constructivismo que permita construir las abstracciones y
conceptos necesarios para programar efectivamente.

\begin{enumerate}
\def\labelenumi{\arabic{enumi}.}
\setcounter{enumi}{10}
\tightlist
\item
  ¿Sugiere como estudiar más adecuadamente una población o fenómeno?
\end{enumerate}

La población de estudio serían estudiantes de telecomunicaciones en
México.

\hypertarget{quuxe9-se-puede-mejorar-en-la-educaciuxf3n}{%
\subsection{Qué se puede mejorar en la
educación}\label{quuxe9-se-puede-mejorar-en-la-educaciuxf3n}}

El sistema de enseñanza no ha cambiado mucho desde la era industrial:
hay un profesor que les enseña la misma clase a todos y se usa un examen
para verificar que memorizaron los conceptos.

En ese entonces no teníamos acceso a la tecnología para verificar hechos
tanto como ahora ni la capacidad de compartir los procesos que usamos
para lograr una tarea determinada. Hoy tenemos la ventaja de que podemos
buscar hechos y procedimientos en internet instantáneamente y seguir la
receta.\footnote{La memorización sigue siendo importante porque para
  generar algo nuevo y no meramente seguir instrucciones es necesario
  entender cómo funcionan las cosas. Los hábitos son aún más importante
  porque nos permiten hacer las cosas en tiempo y forma.

  Parece que la diferencia de productividad entre los mejores
  programadores y los peores es de un orden de magnitud y se debe
  principalmente a los hábitos (McConnell, 2004)}

Las cosas que desde mi perspectiva podrían mejorar en la educación y que
están en nuestras posibilidades tecnológicas:

\begin{itemize}
\item
  Permitir el avance individual de las personas en función de su
  habilidad.
\item
  Un curso práctico con exámenes prácticos, personalizados, generados
  automáticamente y que tienen un semestre para completar.
\item
  Verificar automáticamente la resolución de los problemas.
\item
  Realizar proyectos de aplicación real y no únicamente proyectos
  escolares.
\end{itemize}

\hypertarget{marco-teuxf3rico}{%
\section{Marco teórico}\label{marco-teuxf3rico}}

\hypertarget{desarrollo-del-pensamiento-abstracto-de-acuerdo-al-constructivismo-de-piaget}{%
\subsection{Desarrollo del pensamiento abstracto de acuerdo al
constructivismo de
Piaget}\label{desarrollo-del-pensamiento-abstracto-de-acuerdo-al-constructivismo-de-piaget}}

Piaget et~al. (2016) describe varias etapas en que los niños desarrollan
sus estructuras cognitivas. A lo largo del libro se presenta evidencia
obtenida a partir de experimentos para apoyar sus conclusiones.

Se describe con todo detalle la construcción del conocimiento en niños
de diferentes edades explicando su progresión general y el nivel de
maestría con que lo dominan.

Esa construcción corresponde a la formación de la memoria mediante la
organización de experiencias senso-motores, aplicando la noción de
permanencia, primero de los objetos y luego de sus propiedades
(sustancia, peso, volumen, longitud\ldots), para construir
representaciones mentales cada vez más elaboradas de la realidad
(lugares, episodios, acontecimientos, historias).

Esas representaciones mentales, a su vez, hacen posible la
representación de operaciones con elementos del mundo y permiten
abstraer las propiedades ya no de los objetos, sino de las operaciones
mismas.

En estos procesos, el juego cumple un papel fundamental al poner en
acción:

\begin{itemize}
\tightlist
\item
  conceptos
\item
  conocimientos
\item
  percepciones
\end{itemize}

y contrastar el resultado de las expectativas con la experiencia.

Las etapas de desarrollo que se describen son:

\begin{enumerate}
\def\labelenumi{\arabic{enumi}.}
\item
  Etapa senso-motriz:

  Donde se construyen hábitos a partir de reflejos, se asocian los
  hábitos adquiridos como medios para lograr objetivos y finalmente se
  adquiere la capacidad para generar y refinar esos medios.
\item
  Construcción de lo real:

  Sobre las estructuras desarrolladas en la etapa senso-motriz se
  desarrollan percepciones y nociones que permiten representar
  internamente el mundo que nos rodea e interactuar con él.
\item
  Diferenciación del sujeto:

  En la interacción con el mundo se van diferenciando la percepción de
  la realidad respecto de la percepción de nosotros mismos.

  La percepción propia nos permite también generar la percepción de los
  otros.
\item
  Semiótica:

  A partir de interactuar con el mundo desarrollamos varios lenguajes
  simbólicos (p.e. imagen mental, representación actuada, palabras,
  dibujos) para representar el mundo y nuestra interacción con él.

  Esas representaciones simbólicas nos permiten comunicar a otras
  personas nuestra representación del mundo, y nuestras reacciones
  afectivas. (Piaget elabora acerca del desarrollo de la moral y las
  interacciones sociales pero no elabora acerca de la construcción del
  conocimiento acerca de las propias emociones).
\item
  Desarrollo operatorio:

  Las múltiples representaciones que se desarrollan en las etapas
  anteriores (sensomotoras, propioceptivas, verbales, simbólicas),
  permiten representar nuestra interacción con el mundo y permite
  abstraer las propiedades de las operaciones.

  El proceso ocurre primero en operaciones concretas (sumas, restas,
  traslaciones), pero luego se generaliza a operaciones proposicionales
  donde se separa la forma de las operaciones de su contenido, y eso
  permite razonar acerca de proposiciones ``en las que aún no se cree''
  o en situaciones donde la información es incompleta.
\end{enumerate}

\hypertarget{teoruxeda-de-la-instrucciuxf3n-de-bruner}{%
\subsection{Teoría de la instrucción de
Bruner}\label{teoruxeda-de-la-instrucciuxf3n-de-bruner}}

La instrucción debe fomentar el interés del estudiante, presentar los
conocimientos en un orden adecuado para el aprendizaje, ofrecer
múltiples caminos para alcanzar un objetivo, generar un ambiente seguro
y evaluar en función de los objetivos (Bruner, 1978).

\hypertarget{notas-y-citas}{%
\subsection{Notas y citas}\label{notas-y-citas}}

Las personas aprenden a partir de lo que se conoce en su entorno.
(Bruner, 1978, p. 1)

La percepción se desarrolla generando un modelo del mundo que permite
tomar decisiones acertadas con muy poca información. (Bruner, 1978, p.
2)

\begin{quote}
Teaching is vastly facilitated by the medium of language, which ends by
being not only a medium for exchange but the instrument that the learner
can then use himself in bringing order to the environment. (Bruner,
1978, p. 5)
\end{quote}

Ejemplo acerca de que ``el vaso más lleno'' también puede ser ``el vaso
más vacío'' si tiene más volumen de agua y más volumen sin agua que otro
independientemente de la proporción de agua, que es el centro de
decisión en los adultos.

\begin{center}\rule{0.5\linewidth}{0.5pt}\end{center}

La descripción acerca de la anécdota de ``«getting the luff out of the
main» not making contact with the muscles'' me hizo pensar en que los
centros del lenguaje están relacionados con toda la experiencia sensible
de las personas.

\begin{center}\rule{0.5\linewidth}{0.5pt}\end{center}

\begin{quote}
The nature of intellectual development is that it seems to run the
course of these three systems of representation until the human being is
able to command all three. (Bruner, 1978, p. 12)
\end{quote}

Para explicar la incapacidad de una persona para entender la invariante
de una transformación con respecto de la apariencia, Brunner sugiere que
es el uso del lenguaje lo que permite internalizarlo (Bruner, 1978, pp.
13-14)

Los niños usan el lenguaje para indicar las cosas que están en su
entorno primero y después desarrollan la capacidad de usarlas para
describir lo que no está allí (Bruner, 1978, p. 14)

Si no se dominan habilidades básicas, las habilidades complejas quedan
fuera del alcance. (p.~e. blancos educados vs.~negros salvajes) (Bruner,
1978, p. 29)

Sistemas simbólicos para procesar información (Bruner, 1978, p. 28): -
manipulación y acción - organización perceptual e imágenes - simbólica

\begin{quote}
``John Kemeny, did a survey of high-school mathematics teaching a decade
ago and found no mathematics newer than a hundred years old being
taught!'' (Bruner, 1978, p. 37)
\end{quote}

\begin{quote}
``A theory of instruction is prescriptive in the sense that it sets
forth rules concerning the most effective way of archieving knowledge or
skill.'' (Bruner, 1978, p. 40)
\end{quote}

4 partes de una teoría de instrucción (Bruner, 1978, pp. 40-42): -
experiencias para implantar predisposición a aprender, - estructura
óptima para el aprendizaje, - secuencias efectivas para presentar el
material, - ritmo, premiso y castigos para el proceso.

Es conveniente buscar la forma de convertir las recompensas externas por
recompensas internas pero aún no se entiende cómo (Bruner, 1978, pp.
41-42).

El aprendizaje es un proceso social y el estudiante necesita habilidades
sociales para involucrarse en el proceso (Bruner, 1978, pp. 42-43).

El instructor reduce el peligro de la exploración y aumenta la
posibilidad de explorar alternativas correctas (Bruner, 1978, p. 44).

El problema se puede representar en forma simple y la representación se
puede describir por: modo (acciones, imágenes, símbolos), economía,
poder (Bruner, 1978, pp. 44-45).

La economía puede afectarse por el modo o por la secuencia del material
(Bruner, 1978, p. 46).

Si el intelecto se desarrolló en el este orden: acción → imagen →
símbolo, es posible que la secuencia óptima de aprendizaje sea esa
(Bruner, 1978, p. 49).

Los criterios de la evaluación afectan el orden de la secuencia óptima
de aprendizaje (Bruner, 1978, p. 50).

Para aprender se pueden realizar hipótesis y pruebas, de ese proceso se
pueden desarrollar heurísticas (Bruner, 1978, pp. 51-52). La habilidad
del solucionador depende de su estado interno (Bruner, 1978, p. 52). En
ocasiones, la información negativa puede ayudar a quien aprende (Bruner,
1978, p. 53).

Posiblemente convenga validar experimentalmente el curriculum mientras
se desarrolla (Bruner, 1978, p. 54).

Experimento para enseñar matemáticas con cubos de madera, luego
representaciones gráficas y finalmente representación simbólica (Bruner,
1978, pp. 57-64).

\begin{quote}
``We reached the tentative conclusion that it was probably necessary for
a child, learning math, to have not only a firm sense of the abstraction
underlying what he was working on, but also a good stock of visual
images for embodying them. For without the latter it is difficult to
track correspondences an check what one is doing symbolically.''
(Bruner, 1978, p. 66)
\end{quote}

A pesar de desarrollar la capacidad de abstracción, los alumnos no dejan
ir sus imágenes, que les permiten explorar los problemas que dominan
(Bruner, 1978, p. 68).

Dada la variabilidad individual, no existe una secuencia ideal para
cualquier grupo de estudiantes (Bruner, 1978, p. 71). Un curriculum debe
tener:

\begin{itemize}
\tightlist
\item
  formas diferentes de activar estudiantes,
\item
  varias secuencias,
\item
  oportunidades para saltarse partes.
\end{itemize}

\begin{quote}
``Un curriculum debe tener muchas vías para llegar al mismo resultado
general'' (Bruner, 1978, p. 71)
\end{quote}

\begin{quote}
``Saber es un proceso, no un producto.'' (Bruner, 1978, p. 72)
\end{quote}

Un buen curso contiene su disciplina en formas convenientes para su
aprendizaje (Bruner, 1978, p. 73).

\begin{quote}
``Las cinco grands fuerzas humanizantes son generación de herramientas,
lenguaje, organización social, gestión de la infancia prolongada y la
necesidad de explicar el mundo.'' (Bruner, 1978, p. 75)
\end{quote}

Presentar la lingüística como normativa mata el interés de los niños
(Bruner, 1978, p. 76).

Es más fácil preguntar si ofreces un punto de comparación. (Bruner,
1978, p. 77).

4 técnicas (Bruner, 1978, p. 92):

\begin{itemize}
\tightlist
\item
  Contrastar con algo conocido
\item
  Estimular hipótesis
\item
  Participación: diseñar un juego con las propiedades buscadas.
\item
  Estimular la auto-consciencia.
\end{itemize}

\begin{quote}
``We plan to design materials in which children have an opportunity to
do this sort of thinking with questions related to the course ---
possibly in connection with {[}relevant materials{]}.'' (Bruner, 1978,
p. 95)
\end{quote}

\begin{quote}
``Games go a long way toward getting children involved in understanding
{[}\ldots{]}'' (Bruner, 1978, p. 95)
\end{quote}

\begin{quote}
``Richard Crutchfield has produced results in this sphere {[}stimulating
self-consciousness about thinking{]} using nothing more complicated than
a series of comic books in which the adventures of a detective, aided by
his nephew and niece, are recounted.'' (Bruner, 1978, p. 95)
\end{quote}

La carga de trabajo de los profesores les impide hacer un curso en sus
términos. (Bruner, 1978, p. 97)

Una unidad contiene (Bruner, 1978, p. 98): - charlas con el profesor -
consultas y contrasetes - dispositivos (materiales) - ejercicios modelo
- documentales - material suplementario

\begin{quote}
Las películas producen pasividad (Bruner, 1978, p. 99)
\end{quote}

\begin{quote}
"I have often tought that I would do more for my students by teaching
them to write and think in English than by teaching them my own subject.
(Bruner, 1978, p. 102)
\end{quote}

\begin{quote}
``{[}\ldots{]} words are invitations to form concepts.'' (Bruner, 1978,
p. 105)
\end{quote}

Hay \sout{6} 7 funciones del lenguaje (Bruner, 1978, p. 106;
\textbf{sebeok-1960?}):

\begin{itemize}
\tightlist
\item
  discursiva
\item
  emotiva
\item
  conativa (imperativa)
\item
  referencial
\item
  metalingual
\item
  poética
\item
  fática (ajá, aliento mínimo, mantener contacto)
\end{itemize}

Los niños mayores pueden inhibir funciones entrenadas con mayor
facilidad que los menores, pero cuándo les contaron una historia con
motivos y razones, los niños pequeños pudieron hacer la tarea de la
misma forma. (Bruner, 1978, p. 108).

\begin{quote}
The diagnosis of gushiness carries no remeedial prescription with it. I
stumbled on the happy formula. Could she rewrite the piece without a
single adjective, not a one? (Bruner, 1978, p. 110).
\end{quote}

Es posible que escribir sea una generalización de segundo orden, como el
álgebra a la aritmética. (Bruner, 1978, p. 111).

\begin{quote}
``It was Dante, I believe, who commented thata the poor workman hated
his tools. It is more than a little troubling to me that so many of our
students dislike two of the major tools of thought ---mathematics and
the conscious deployment of their native language in written form, both
devices for ordering thoughts about things and thoughts about thoughts.
I should hope that in the new era that lies ahead we will give a proper
consideration to making these tools more lovable. Perhaps the best way
to make them so is to make them more powerful in the hands of their
users.'' (Bruner, 1978, p. 112)
\end{quote}

Las limitaciones de los humanos les ponen en peligro mortal si
necesitaran reinventar su cultura. (Bruner, 1978, pp. 113-114)

\begin{quote}
``Donald Hebb, has suggested that the child is drinking in the world,
better to construct his neural''models" of the environment." (Bruner,
1978, p. 115)
\end{quote}

\begin{quote}
"Little enough is known about how to help a child become master of his
own attention, \textless{} to sustain it over a long, connected
sequence. (Bruner, 1978, p. 116)
\end{quote}

Aunque no se sabe mucho acerca de cómo instruir a los estudiantes a
dirigir la atención, se pueden hacer actividades interesantes para
mantenerla. (Bruner, 1978, pp. 115-116)

El interés está ligado a la competencia. {[}Bruner (1978) 118--119; {]}

Puede estimularse el término de ciertas tareas complejas (y
planificadas) con interrupciones. Las tareas simples se benefician de
las interrupciones. (Bruner, 1978, p. 120)

Los valores que promueve la sociedad, forman incentivos de
comportamiento que las personas pueden aprender. (Bruner, 1978, p. 124)

El comportamiento de las personas es más variable con respecto de la
situación que con respecto de la persona. (Bruner, 1978, p. 125;
\textbf{barker-1963?})

Las personas tienen la motivación inherente de aprender pero en las
escuelas se convierte en un problema más específico cuando el currículum
está definido. (Bruner, 1978, p. 127)

Comparando niños con problemas de aprendizaje pero sin disfunción
aparente, encontraron que los niños aprendían constantemente, pero de
una forma en que el conocimiento no se transfería a otras actividades.
(Bruner, 1978, p. 131) Otro problema de ellos es que buscaban defenderse
de su entorno, como lidiar con un profesor hostil. (Bruner, 1978, p.
132)

Las personas desarrollan su pensamiento por medio de la acción. (Bruner,
1978, pp. 133-134)

\begin{quote}
``Pero hay cosas que son tan importantes que uno sólo puede bromear al
respecto'' ---Niels Bohr, citado en (Bruner, 1978, p. 135).
\end{quote}

Un ambiente psicológico seguro es importante para el aprendizaje
(Bruner, 1978, pp. 129-148).

La evaluación es efectiva si (Bruner, 1978, pp. 163-168): - Se usa para
guiar la construcción del curriculum y su pedagogía. - Combina el
esfuerzo del estudiante y el evaluador. - Es un esfuerzo conjunto con
todos a-bordo. - Se usa para mejorar el proceso en su conjunto. - Se usa
para desarrollar habilidades intelectuales. - Considera al profesor y el
estudiante. - Contribuye a la teoría de la instrucción.

Para transmitir un curso es necesario que el estudiante tenga capacidad
de lectura crítica: razonar los argumentos de los textos. (Bruner, 1978,
pp. 168-171)

\hypertarget{suxedntesis}{%
\subsection{Síntesis}\label{suxedntesis}}

Se puede enseñar cualquier cosa a casi cualquier niño en cualquier etapa
de desarrollo si se estructuran los conceptos en los términos que maneja
el estudiante, y de forma que promuevan el descubrimiento y
generalización de la estructura subyacente de la realidad (Bruner,
1977).

Si eso se hace bien, los estudiantes desarrollan la seguridad para
desarrollar intuiciones que son más rápidas que los procesos formales,
pero deben validarse (Bruner, 1977, ch.~4).

Para transmitir la estructura subyacente de la realidad pueden usarse
materiales de apoyo (Bruner, 1977, ch.~1,6).

\hypertarget{citas}{%
\subsection{Citas}\label{citas}}

Visión estructuralista e intuicionista.

Referentes: Piaget, Chomsky, Lévi-Strauss.

Siempre se puede partir del conocimiento del estudiante para llevarlo un
paso más cerca de un concepto (Bruner, 1977, pp. ix-x).

\begin{quote}
El curriculum es más para los profesores que para los estudiantes. Si no
puede cambiar, mover, perturbar, informar a los profesores, no tendrá
efecto en quienes ellos enseñan. (Bruner, 1977, p. xv).
\end{quote}

El curriculum no tiene forma de evitar la influencia del profesor.
(Bruner, 1977, p. xv)

\begin{quote}
``Cada generación le da una nueva forma a las aspiraciones que dan forma
a la educación de su tiempo.'' (Bruner, 1977, p. 1)
\end{quote}

\begin{quote}
``¿Qué deberíamos estudiar y para qué fin?'' (Bruner, 1977, p. 1)
\end{quote}

No enseñar las cosas de la forma correcta y cambiar la currícula sin
sentido nos ha privado de los conocimientos de frontera (Bruner, 1977,
pp. 3-4).

El problema de la transferencia es interesante y su estudio revela que
las prácticas de enseñanza no son tan efectivas (Bruner, 1977, pp. 5-6).
Para la transferencia es más importante si sabe usar las habilidades que
si conoces el nombre de las habilidades (Bruner, 1977, p. 8).

Los requerimientos de productividad (tantos profesionistas de cada tipo)
impulsan a la diversidad y los costos impulsan a la uniformidad.
(Bruner, 1977, p. 9)

\begin{quote}
"It is plain that, in general, scientific and mathematical aptitudes can
be discovered earlier than other intellectual talents. Ideally, schools
should allow students to go ahead in different subjects as rapidly as
they can. But the administrative problems that are raised when one makes
such an arrangement possible are almost inevitably beyond the resources
that schools have available for dealing with them. The answer will
probably lie in some modification or abolition of the system of grade
levels in some subjects, notably mathematics, along with a program of
course enrichment in other subjects. ---(Bruner, 1977, pp. 10-11)
\end{quote}

``Aprender la estructura en vez de hechos y técnicas está en el centro
del problema de la transferencia.'' (Bruner, 1977, p. 12)

Si el primer aprendizaje debe hacer más sencillo el aprendizaje
posterior, debe hacerlo proveyendo una imagen general de las cosas en
términos tales que la relación entre las cosas que se encuentran antes y
después sea tan clara como sea posible. (Bruner, 1977, p. 12)

\begin{quote}
Los fundamentos de cualquier asignatura se pueden enseñar a cualquier
edad en alguna forma. (Bruner, 1977, p. 12)
\end{quote}

4 temas (Bruner, 1977, pp. 10-16):

\begin{itemize}
\tightlist
\item
  El rol de la estructura en el aprendizaje.
\item
  La preparación para aprender.
\item
  La naturaleza de la intuición intelectual.
\item
  El deseo de aprender y cómo estimularlo.
\end{itemize}

La premisa: la actividad intelectual siempre es la misma, y la
diferencia entre las actividades del niño aprendiendo y el científico en
el laboratorio son de grado y no de especie (Bruner, 1977, p. 14).

Además de ordenar los temas, los profesores guían a los estudiantes por
caminos que fomentan el descubrimiento (Bruner, 1977, p. 20−24).

Las cosas que no se ordenan en una estructura se olvidan fácilmente
(Bruner, 1977, p. 24)

\begin{quote}
A good theory is the vehicle not only for understanding a phenomenon now
but also for remembering tomorrow. (Bruner, 1977, p. 25)
\end{quote}

La evaluación puede diseñarse para fomentar el aprendizaje (Bruner,
1977, p. 30)

La ruta de estudio debe determinarse por el entendimiento más
fundamental de la asignatura y los principios básicos le deben dar
estructura. (Bruner, 1977, p. 31)

\begin{quote}
«Any subject can be taught effectively in some intellectually honest
form to any child at any stage of development. It is a bold hypothesis
{[}\ldots{]}. No evidence exist to contradict it; considerable evidence
is being amassed that supports it. (Bruner, 1977, p. 33)
\end{quote}

\begin{quote}
A cualquier etapa del desarrollo el niño tiene una forma particular de
ver el mundo y explicárselo a sí mismo. (Bruner, 1977, p. 33)
\end{quote}

\begin{quote}
Cualquier idea puede representarse honestamente y der forma útil en las
formas de pensamiento de los niños de edad escolar, y esas primeras
representaciones pueden hacerse más poderosas y precisas más fácilmente
por ese primer aprendizaje. (Bruner, 1977, p. 33)
\end{quote}

3 etapas Piaget: - Etapa pre-operativa: establecer relaciones enter
experiencia y acción → manipular el mundo a través de la acción desde e
desarrollo de lenguaje hasta la manipulación simbólica. - Etapa de
operaciones concretas: - Etapa de operaciones simbólicas.

\begin{quote}
It is easy to ask trivial questions or to lead the child to ask trivial
questions. It is also easy to ask impossibly difficult questions. The
trick is to find the medium questions that can be answered and that take
you somewhere. This is the big job of teachers and textbooks (Bruner,
1977, p. 40).
\end{quote}

\begin{quote}
The most elemental forms of reasoning {[}\ldots{]} rest on the principle
of the invariance of quantities. Grasping the idea of invariance is
beset with difficulties for the child, often unsuspected by teachers
(Bruner, 1977, p. 42).
\end{quote}

El orden óptimo para enseñar la materia no es necesariamente el orden
histórico de descubrimiento. Puede que las ideas necesarias para
entender algo se hayan formalizado recientemente (Bruner, 1977, p.
42---46).

Las herramientas formales deben introducirse después del entendimiento
intuitivo (Bruner, 1977, p. 46).

Si se enseña con una estructura apropiada, se puede enseñar a niños
menores de lo que la instrucción normalmente permite (Bruner, 1977, p.
44).

Enseñar a aprender se ha mostrado efectivo para facilitar la
recuperación de monos después de daño cerebral inducido. (Bruner, 1977,
p. 47)

El aprendizaje se divide en 3 etapas: - adquisición - transformación -
evaluación (Bruner, 1977, p. 48)

Un buen episodio de aprendizaje reflexiona en el conocimiento previo y
permite generalizar más allá del episodio. (Bruner, 1977, p. 49)

Para ajustar el aprendizaje al estudiante, - disminuir o aumentar la
longitud del episodio - agregar recompensas extrínsecas - dramatizar el
reconocimiento (Bruner, 1977, p. 49)

¿Cómo se balancean las recompensas extrínsecas e intrínsecas? (Bruner,
1977, p. 50)

Una propuesta para la enseñanza por descubrimiento es dar un conjunto
mínimo de hechos y un ejercicio que enfatice cómo ir más allá. (Bruner,
1977, pp. 50-51)

Se propone un curriculum en espiral: Transformar el material a formas
lógicas que tienten a los estudiantes a avanzar. Eliminar el material
que no los hace mejores adultos. Desarrollar y volver a desarrollar los
temas, cada vez de forma más elaborada. (Bruner, 1977, p. 52)

→ En el libro «Dive into python» recuerdo que algún capítulo decía «de
hecho mentí» y reintroducía el concepto nuevamente con matices y
complejidad añadida.

\begin{center}\rule{0.5\linewidth}{0.5pt}\end{center}

Aunque las escuelas se enfocan en las habilidades formales, esas
dependen del desarrollo de la intuición. (Bruner, 1977, ch.~4, p.~57)

El pensamiento formal es secuencial, y con reglas específicas. El
pensamiento intuitivo llega a una respuesta (correcta o incorrecta) sin
consciencia acerca de cómo se llegó a esa conclusión. (Bruner, 1977,
ch.~4, p.~57--58)

Debe respetarse el pensamiento intuitivo como una fuente de hipótesis.
Cuando sea posible las respuestas deben verificarse por métodos
analíticos (o mediciones). (Bruner, 1977, ch.~4, p.~58--62)

Para desarollar la intuición, se propone el método de intentar adivinar
la respuesta y luego analizarla. (Bruner, 1977, pp. ch.4, p.62)

La intuición da lugar a métodos heurísticos. Estos métodos pueden dar
respuestas cuando no hay algoritmos disponibles y suelen ser más rápidos
que los métodos analíticos. (Bruner, 1977, ch.~4, p.64--65)

La intuición se favorece desarrollando la auto-confianza del estudiante.
(Bruner, 1977, ch.~4, p.65)

Los estudiantes intentan usar métodos procedimentales cuando hay más
incertidumbre aunque no sean apropiados (Bruner, 1977, ch.~4, p.64--65).

Para diseñar el \emph{curriculum} pueden usarse los motivos para el
aprendizaje y los objetivos que es estudiante espera lograr,
distinguiendo entre los objetivos de largo plazo y los pasos para llegar
allí (Bruner, 1977, ch.~5, p.~69).

La motivación es esencial para la excelencia (Bruner, 1977, ch.~5,
p.~69).

Se debe impedir que los motivos para aprender se vuelvan pasivos, deben
basarse tanto como sea posible en expandir el interés (Bruner, 1977,
ch.~5, p.~80).

Las ayudas para el aprendizaje, pueden ser dispositivo para la
experiencia indirecta (Bruner, 1977, ch.~6, p.~81).

También pueden ayudar al estudiante a interiorizar la estructura del
fenómeno (Bruner, 1977, ch.~6, p.~81).

Enseñar es análogo a programar y varios programadores buenos son
profesores (Bruner, 1977, ch.~6, p.~83--84).

Las computadoras no deshumanizarán más el aprendizaje de lo que los
libros lo hicieron antes (Bruner, 1977, ch.~6, p.~84).

Se ha intentado automatizar el aprendizaje con videos en vez de
profesores (Bruner, 1977, ch.~6, p.~85).

Comunicar el conocimiento depende de la maestría en ese conocimiento
(Bruner, 1977, ch.~6, p.~88).

La enseñanza básica tiene requisitos más amplios que otros niveles que
no están bien enetendidos (Bruner, 1977, ch.~6, p.~90).

\hypertarget{teoruxeda-del-horizonte-proximal-del-conocimiento-vigotsky}{%
\subsection{Teoría del horizonte proximal del conocimiento
Vigotsky}\label{teoruxeda-del-horizonte-proximal-del-conocimiento-vigotsky}}

La teoría de Vigotsky surge cuando el empiricismo y el racionalismo se
disputaban como teoría epistemológica dominante en Europa (Vygotski
et~al., 2009, p. 18).

En contraposición a Piaget, que establece que hay etapas bien definidas
para la construcción del conocimiento, Vigotsky establece que el orden
en que se construye el conocimiento, depende de la interacción social y
la necesidad.

\sout{Establece que los conocimientos se adquieren en función de las
necesidades que tenemos y por medio de la interacción social.}

Se puede interpretar como que tenemos elementos para ir construyendo el
conocimiento, y que vamos generando puentes para alcanzar los nuevos
conocimientos partiendo de nuestros conocimientos actuales.

Sin embargo, existen conocimientos que están más allá de lo que podemos
entender ahora, porque no se han desarrollado los conocimientos que
necesitan conectarse para adquirir esos nuevos conocimientos.

\hypertarget{aprendizaje-por-descubrimiento}{%
\subsection{Aprendizaje por
descubrimiento}\label{aprendizaje-por-descubrimiento}}

\hypertarget{conceptos-aplicables-en-ciberseguridad}{%
\subsection{Conceptos aplicables en
ciberseguridad}\label{conceptos-aplicables-en-ciberseguridad}}

\hypertarget{buenas-pruxe1cticas-de-ciberseguridad}{%
\subsubsection{Buenas prácticas de
ciberseguridad}\label{buenas-pruxe1cticas-de-ciberseguridad}}

Lehman (1980) explora cómo se desarrollan los programas desde su
planeación hasta su construcción. Establece que es necesaria una
disciplina para llevar a cabo los proyectos.

Existen un par de prácticas sencillas que pueden mejorar la seguridad de
los programas:

\begin{itemize}
\tightlist
\item
  Validar las entradas de información
\item
  Verificar las condiciones de ejecución
\end{itemize}

\hypertarget{estilo-de-cuxf3digo}{%
\subsubsection{Estilo de código}\label{estilo-de-cuxf3digo}}

El estilo de código es independiente de la capacidad del código para
resolver el problema esperado, y no suele hacer diferencia en cuanto al
código de máquina que se genera a partir de él o la eficiencia con que
se ejecuta el programa.

El estilo de código es una serie de convenciones acerca de:

\begin{itemize}
\tightlist
\item
  Cómo se nombran los elementos del código.
\item
  Cómo se usa la indentación para aumentar la legibilidad.
\end{itemize}

Estas prácticas mejoran la mantenibilidad del código; que puede ayudar a
mantener los programas libres de errores lógicos y de seguridad.

\hypertarget{diferencia-entre-buenos-programadores}{%
\subsubsection{Diferencia entre buenos
programadores}\label{diferencia-entre-buenos-programadores}}

La principal diferencia entre los buenos programadores y los malos
programadores, es el desarrollo de buenos hábitos al momento de generar
código. (McConnell, 2004, p. 833)

Es muy difícil introducir hábitos como la preocupación acerca de
(McConnell, 2004, p. 833):

\begin{itemize}
\tightlist
\item
  Legibilidad del código
\item
  Programación defensiva
\item
  Equilibrio entre rendimiento y legibilidad
\end{itemize}

\hypertarget{integraciuxf3n-continua}{%
\subsubsection{Integración continua}\label{integraciuxf3n-continua}}

Es una serie de metodologías en desarrollo de software que consisten en
automatizar las tareas necesarias para distribuir el software, de manera
que no se requiera invertir trabajo en el despliegue de software.

Bien aplicada, permite a los programadores concentrarse en generar valor
para el negocio, dado que las tareas rutinarias quedan automatizadas.

También se puede implementar de forma que permita detectar problemas en
el código desde que se introducen, reduciendo los tiempos de desarrollo
y los costos por depuración.

Para obtener estos beneficios, debe acoplarse con el Desarrollo basado e
pruebas.

\hypertarget{desarrollo-basado-en-pruebas}{%
\subsubsection{Desarrollo Basado en
Pruebas}\label{desarrollo-basado-en-pruebas}}

Esta metodología podría «prevenir la mayoría de las fallas críticas»
(s.~f., p. 1) pero es una metodología difícil de aprender porque
involucra diseñar simultáneamente dos sistemas divergentes:

\begin{itemize}
\tightlist
\item
  pruebas específicas, y
\item
  código abstracto.
\end{itemize}

La mayoría de los errores de software podría encontrarse en condiciones
de error que no se verifican (s.~f.).

\hypertarget{mantenibilidad-del-cuxf3digo}{%
\subsubsection{Mantenibilidad del
código}\label{mantenibilidad-del-cuxf3digo}}

El software se escribe una vez y luego está en mantenimiento durante un
largo tiempo.

Dado que el mantenimiento es la mayor parte del ciclo de vida del
software, mientras más fácil de mantener sea el código, mejores son las
probabilidades del software para triunfar en el mercado.

\hypertarget{enseuxf1anza-de-ciberseguridad}{%
\subsection{Enseñanza de
ciberseguridad}\label{enseuxf1anza-de-ciberseguridad}}

\hypertarget{pruxe1cticas-comunes-en-la-enseuxf1anza-de-ciberseguridad}{%
\subsubsection{Prácticas comunes en la enseñanza de
ciberseguridad}\label{pruxe1cticas-comunes-en-la-enseuxf1anza-de-ciberseguridad}}

Zinovieva et~al. (2021) recomienda el uso de «Simuladores de
Programación en Línea».

\hypertarget{pruxe1cticas-efectivas-en-la-enseuxf1anza-de-educaciuxf3n}{%
\subsubsection{Prácticas efectivas en la enseñanza de
educación}\label{pruxe1cticas-efectivas-en-la-enseuxf1anza-de-educaciuxf3n}}

\newpage

\hypertarget{marco-regulatorio}{%
\section{Marco regulatorio}\label{marco-regulatorio}}

De acuerdo a (s.~f.), \emph{Marco legal educativo de los Estados Unidos
Mexicanos} (s.~f.)/, \emph{UAJyT} (s.~f.), (s.~f.) el marco legislativo
de la educación está conformado por:

\begin{itemize}
\item
  Art. 2 Constitucional: debido a la composición pluricultural del país,
  la educación también debe ser multicultural
\item
  Art. 3 Constitucional: Derecho a la educación

  De acuerdo al artículo 3o Constitucional, el Estado (Federación,
  Estados, Ciudad de México y Municipios) impartirá y garantizará la
  educación inicial, preescolar, primaria, secundaria, media superior y
  superior, que son los niveles que comprendidos dentro del Sistema
  Educativo Nacional.
\item
  Art. 4: Derecho de los niños
\item
  Art. 5: Derecho a dedicarse a una profesion o trabajo.
\item
  Art. 31 Constitucional: Obligación de los padres para llevar a los
  niños a la escuela
\item
  Art. 73. Constitucional
\item
  Art. 123 Constitucional: describe los derechos de los profesores.
\end{itemize}

Además, las siguientes leyes regulan la educación:

\begin{itemize}
\item
  Art. 9. bis de la Ley de Ciencia y Tecnología
\item
  Ley General de Educación de los Estados Unidos Mexicanos
\item
  Ley Reglamentaria del Artículo 3o. de la Constitución Política de los
  Estados Unidos Mexicanos, en materia de Mejora Continua de la
  Educación
\item
  Ley para la Coordinación de la Educación Superior
\item
  Ley General del Sistema para la Carrera de las Maestras y los Maestros
\item
  Reglamento Interior de la Secretaría de Educación Pública
\item
  Ley General de Educación: reglamentaria del artículo 3º constitucional
  (D.O.F. 30 de octubre de 2019).
\item
  Ley General de Educación Superior (D.O.F. 20 de abril de 2021).
\item
  Ley Reglamentaria del Artículo 3o. de la Constitución Política de los
  Estados Unidos Mexicanos, en materia de Mejora Continua de la
  Educación (D.O.F. 30 de octubre de 2019).
\item
  Ley de Planeación (D.O.F. 5 de enero de 1983).
\item
  Ley de profesiones. Hoy titulada ``Ley Reglamentaria del Artículo 5o.
  Constitucional, relativo al ejercicio de las profesiones en la Ciudad
  de México'' (D.O.F. 26 de mayo de 1945).
\item
  Ley General de los Derechos de Niñas, Niños y Adolescentes (D.O.F. 4
  de diciembre de 2014).
\item
  Ley Federal del Trabajo (D.O.F. 1 de abril de 1970).
\item
  Ley Federal de los Trabajadores al Servicio del Estado, Reglamentaria
  del Apartado B) del Artículo 123 Constitucional (D.O.F. 28 de
  diciembre de 1963).
\item
  Ley General de Responsabilidades Administrativas (D.O.F. 18 de julio
  de 2016)
\end{itemize}

Reglamentos públicos:

Reglamento interior de la Secretaría de Educación Pública (D.O.F. 15 de
septiembre de 2020).

\begin{enumerate}
\def\labelenumi{\alph{enumi})}
\item
  Normas constitucionales:

  \begin{enumerate}
  \def\labelenumii{\alph{enumii})}
  \item
    directamente relacionadas con la función educativa. Tal es el caso
    de los siguientes artículos constitucionales: 3o.; 31; 73, fracción
    XXV; 123, fracción XII, y 130.
  \item
    indirectamente relacionadas con la función educativa, pero cuya
    aplicación reglamentaria tiene una intervención importante en la
    planificación y administración del sector educativo público. Tal es
    el caso de los artículos constitucionales siguientes: 25, 26, 89, 90
    y 123, apartados A y B.
  \end{enumerate}
\item
  Normas sustantivas de la educación. Incluimos en este grupo aquellos
  ordenamientos que regulan directamente la función educativa pública,
  como es el caso de:

  \begin{enumerate}
  \def\labelenumii{\arabic{enumii}.}
  \item
    Ley General de Educación, reglamentaria del artículo 3º
    constitucional (D.O.F. 30 de octubre de 2019).

    De conformidad al artículo 4 de la Ley General de Educación son
    autoridades:

    \begin{enumerate}
    \def\labelenumiii{\arabic{enumiii})}
    \tightlist
    \item
      Autoridad educativa federal o Secretaría, a la Secretaría de
      Educación Pública de la Administración Pública Federal;
    \item
      Autoridad educativa de los Estados y de la Ciudad de México, al
      ejecutivo de cada una de estas entidades federativas, así como a
      las instancias que, en su caso, establezcan para el ejercicio de
      la función social educativa;
    \item
      Autoridad educativa municipal, al Ayuntamiento de cada Municipio;
    \item
      Autoridades escolares, al personal que lleva a cabo funciones de
      dirección o supervisión en los sectores, zonas o centros
      escolares, y
    \item
      Estado, a la Federación, los Estados, la Ciudad de México y los
      municipios.
    \end{enumerate}

    El artículo 7 de la Ley General de Educación reconoce a la educación
    privada, ya que señala que la educación impartida por los
    particulares con autorización o con reconocimiento de validez
    oficial de estudios, se sujetará a lo previsto en la fracción VI del
    artículo 3o. de la Constitución Política de los Estados Unidos
    Mexicanos y al Título Décimo Primero de esta Ley.

    El capítulo III del título segundo habla acerca de los criterios de
    la educación.

    El título III y capítulo IV habla acerca de la educación superior.

    El artículo 22 de la LGE establece la necesidad de evaluación del
    aprendizaje y acreditación; que de acuerdo al Artículo 29, seción IV
    deben estar en acuerdo con los planes de estudios.

    De acuerdo al artículo 37 de la Ley General de Educación, la
    educación básica está compuesta por el nivel inicial, preescolar,
    primaria y secundaria. Los servicios que comprende este tipo de
    educación, entre otros:

    \begin{itemize}
    \tightlist
    \item
      Inicial escolarizada y no escolarizada;
    \item
      Preescolar general, indígena y comunitario;
    \item
      Primaria general, indígena y comunitaria;
    \item
      Secundaria, entre las que se encuentran la general, técnica,
      comunitaria o las modalidades regionales autorizadas por la
      Secretaría;
    \item
      Secundaria para trabajadores, y
    \item
      Telesecundaria.
    \end{itemize}

    De acuerdo a los artículos 44 y 45, a educación media superior
    comprende los niveles de bachillerato, de profesional técnico
    bachiller y los equivalentes a éste, así como la educación
    profesional que no requiere bachillerato o sus equivalentes. Se
    organizará a través de un sistema que establezca un marco curricular
    común a nivel nacional y garantice el reconocimiento de estudios
    entre las opciones que ofrece este tipo educativo. Los servicios que
    comprende este tipo de educación son:

    \begin{itemize}
    \tightlist
    \item
      Bachillerato General;
    \item
      Bachillerato Tecnológico;
    \item
      Bachillerato Intercultural;
    \item
      Bachillerato Artístico;
    \item
      Profesional técnico bachiller;
    \item
      Telebachillerato comunitario;
    \item
      Educación media superior a distancia, y
    \item
      Tecnólogo.
    \end{itemize}

    La evaluación también debe tener como objetivo el diagnóstico para
    idenificar estudiantes sobresalientes y dar atención a quiénes la
    necesitan, de acuerdo al Artículo 67.
  \item
    Ley para la Coordinación de la Educación Superior, D.O. 29-XII-1978;
  \item
    Ley Nacional de Educación para Adultos, D.O. 31-XII-1975;
  \item
    Ley del Ahorro Escolar, D.O. 7-IX-1945;
  \item
    Reglamento de la Ley del Ahorro Escolar, D.O. 8-VI-1946;
  \item
    Ley Federal de Reforma Agraria (capítulo cuarto, artículos 101 y
    102);
  \item
    Reglamento de la Parcela Escolar, D.O. 10-XI-1944;
  \item
    Ley General de Sociedades Cooperativas (artículo 13), D.O.;
  \item
    Reglamento de Cooperativas Escolares, D.O. 16- III-1962;
  \item
    Ley que establece la Educación Normal para los profesores de centros
    de capacitación para el trabajo, D.O. 20-XII-1963;
  \item
    Reglamento para la constitución y funcionamiento de las asociaciones
    de padres de familia en las escuelas dependientes de la Secretaría
    de Educación Pública, D.O. 22-I-1949.
  \item
    Ley General de Planeación

    \begin{quote}
    De acuerdo a la Ley de Planeación y a la Ley General de Educación,
    que señala en el artículo 123 cómo serán los planes, lineamientos,
    evaluación y retroalimentación:

    En las escuelas de educación básica y media superior, la Secretaría
    emitirá los lineamientos que deberán seguir las autoridades
    educativas locales y municipales para formular los programas de
    fortalecimiento de las capacidades de administración escolar, mismos
    que tendrán como objetivos: I. Usar los resultados de la evaluación
    como retroalimentación para la mejora continua en cada ciclo
    escolar;

    \begin{enumerate}
    \def\labelenumiii{\Roman{enumiii}.}
    \setcounter{enumiii}{1}
    \tightlist
    \item
      Desarrollar una planeación anual de actividades, con metas
      verificables y puestas en conocimiento de la autoridad y la
      comunidad escolar, y
    \item
      Administrar en forma transparente y eficiente los recursos que
      reciba para mejorar su infraestructura, comprar materiales
      educativos, resolver problemas de operación básicos y propiciar
      condiciones de participación para que alumnos, maestras, maestros,
      madres y padres de familia o tutores, bajo el liderazgo del
      director, se involucren en la resolución de los retos que cada
      escuela enfrenta.
    \end{enumerate}
    \end{quote}
  \end{enumerate}
\item
  Leyes orgánicas y decretos que crean instituciones educativas.

  Dado que la institución en la que se realiza el proyecto es la
  Universidad Iberoamericana, se usará el
  \href{https://ibero.mx/sites/all/themes/ibero/descargables/corpus/reglamento_de_estudios_de_licenciatura.pdf}{Reglamento
  de Estudios de Licenciatura de la Universidad Iberoamericana}, que
  dedica el «Título tercero Evaluaciones» a este tema, estableciendo
  que:

  \begin{quote}
  Las prácticas evaluatorias son parte del proceso universitario y
  tienen por objeto comparar los logros del aprendizaje del alumno con
  los objetivos del programa, de sus áreas y de una parte o de la
  totalidad de los cursos que lo conforman.

  Las evaluaciones pueden realizarse antes, durante o después de un
  proceso de aprendizaje.

  Los sistemas de evaluación deben ser diseñados de manera que:

  \begin{enumerate}
  \def\labelenumii{\alph{enumii})}
  \tightlist
  \item
    La Universidad pueda comprobar el logro de los objetivos de
    aprendizaje y dar testimonio de la preparación humana y académica de
    sus egresados.
  \item
    El alumno tenga la oportunidad de conocer sus logros.
  \item
    El alumno se sienta motivado hacia el estudio e incremente su
    interés al tener la certeza de los avances que realiza.
  \item
    Los profesores y los alumnos puedan comprobar la eficiencia de los
    métodos pedagógicos para alcanzar las metas universitarias y los
    objetivos específicos de los programas en cada una de las etapas.
  \end{enumerate}

  CAPÍTULO III Evaluación ordinaria para acreditar una materia

  Artículo 36 La evaluación ordinaria para acreditar una materia tiene
  lugar en el curso lectivo, preferentemente a todo lo largo del mismo,
  y consiste en una comparación entre el aprendizaje obtenido por el
  alumno y los objetivos de la materia. La evaluación ordinaria puede
  llevarse a cabo mediante exámenes parciales, la presentación de
  proyectos o trabajos, realización de prácticas de campo, reportes de
  laboratorios o talles, seminarios, examen global, evaluaciones
  departamentales u otras formas aprobadas por el Consejo Técnico del
  programa respectivo.

  Deben realizarse al menos tres evaluaciones durante el curso
  utilizando la técnica y los instrumentos que se consideren más
  apropiados para verificar el aprendizaje. En las materias del Área de
  Síntesis y Evaluación se deberá dar especial atención a este proceso.

  Los resultados de la primera evaluación deberán ser entregados antes
  del periodo de bajas académicas establecido por la Dirección de
  Servicios Escolares.

  Para acreditar una materia por medio de una evaluación ordinaria es
  necesario estar inscrito en ella o en el periodo correspondiente y no
  haber faltado injustificadamente a más del 20 por ciento de las
  sesiones programadas de acuerdo al calendario escolar.
  \end{quote}
\end{enumerate}

El acuerdo 648 establece las normas generales para la evaluación,
acreditación, promoción y certificación en la educación básica.

Entre las Normas Oficiales Mexicanas sólo encontré regulación acerca de
la educación en materia de salud. Pero no parece haber alguna norma que
hable acerca de la Educación.

\href{http://www.scielo.org.mx/scielo.php?script=sci_arttext\&pid=S0185-26982013000500002}{Legislación
en educación superior}

\href{http://dof.gob.mx/nota_detalle.php?codigo=5618145\&fecha=12/05/2021}{SERVICIOS
EDUCATIVOS- DISPOSICIONES A LAS QUE SE SUJETARÁN AQUELLOS PARTICULARES
QUE PRESTEN SERVICIOS EN LA MATERIA}

\href{http://www.scielo.org.mx/scielo.php?script=sci_arttext\&pid=S0185-26982000000100004}{Acerca
de la normatividad internacional en materia de educación}

https://www.scjn.gob.mx/normativa-nacional-internacional

Además existen acuerdos internacionales

\hypertarget{acerca-de-los-fines-de-la-educaciuxf3n}{%
\subsection{Acerca de los fines de la
educación}\label{acerca-de-los-fines-de-la-educaciuxf3n}}

\begin{quote}
El artículo 15 de la Ley General de Educación establece que la educación
que imparta el Estado, sus organismos descentralizados y los
particulares con autorización o con reconocimiento de validez oficial de
estudios, persigue los siguientes fines: I. Contribuir al desarrollo
integral y permanente de los educandos, para que ejerzan de manera plena
sus capacidades, a través de la mejora continua del Sistema Educativo
Nacional; II. Promover el respeto irrestricto de la dignidad humana,
como valor fundamental e inalterable de la persona y de la sociedad, a
partir de una formación humanista que contribuya a la mejor convivencia
social en un marco de respeto por los derechos de todas las personas y
la integridad de las familias, el aprecio por la diversidad y la
corresponsabilidad con el interés general; III. Inculcar el enfoque de
derechos humanos y de igualdad sustantiva, y promover el conocimiento,
respeto, disfrute y ejercicio de todos los derechos, con el mismo trato
y oportunidades para las personas; IV. Fomentar el amor a la Patria, el
aprecio por sus culturas, el conocimiento de su historia y el compromiso
con los valores, símbolos patrios y las instituciones nacionales; V.
Formar a los educandos en la cultura de la paz, el respeto, la
tolerancia, los valores democráticos que favorezcan el diálogo
constructivo, la solidaridad y la búsqueda de acuerdos que permitan la
solución no violenta de conflictos y la convivencia en un marco de
respeto a las diferencias; VI. Propiciar actitudes solidarias en el
ámbito internacional, en la independencia y en la justicia para
fortalecer el ejercicio de los derechos de todas las personas, el
cumplimiento de sus obligaciones y el respeto entre las naciones; VII.
Promover la comprensión, el aprecio, el conocimiento y enseñanza de la
pluralidad étnica, cultural y lingüística de la nación, el diálogo e
intercambio intercultural sobre la base de equidad y respeto mutuo; así
como la valoración de las tradiciones y particularidades culturales de
las diversas regiones del país; VIII. Inculcar el respeto por la
naturaleza, a través de la generación de capacidades y habilidades que
aseguren el manejo integral, la conservación y el aprovechamiento de los
recursos naturales, el desarrollo sostenible y la resiliencia frente al
cambio climático; IX. Fomentar la honestidad, el civismo y los valores
necesarios para transformar la vida pública del país, y X. Todos
aquellos que contribuyan al bienestar y desarrollo del país.
\end{quote}

\hypertarget{acerca-de-la-calidad-de-la-educaciuxf3n}{%
\subsection{Acerca de la calidad de la
educación}\label{acerca-de-la-calidad-de-la-educaciuxf3n}}

En el \href{https://www.oecd.org/education/school/46216786.pdf}{Acuerdo
de cooperación México-OCDE para mejorar la calidad de la educación de
las escuelas mexicanas} se establece la evaluación como medio para
mejorar la calidad educativa.

\begin{quote}
De acuerdo a los artículos 107 y 108 de la Ley General de Educación, las
autoridades educativas, en el ámbito de sus respectivas competencias,
emitirán una Guía Operativa para la Organización y Funcionamiento de los
Servicios de Educación Básica y Media Superior, el cual será un
documento de carácter operativo y normativo que tendrá la finalidad de
apoyar la planeación, organización y ejecución de las actividades
docentes, pedagógicas, directivas, administrativas y de supervisión de
cada plantel educativo enfocadas a la mejora escolar, atendiendo al
contexto regional de la prestación de los servicios educativos. Para el
proceso de mejora escolar, se constituirán Consejos Técnicos Escolares
en los tipos de educación básica y media superior, como órganos
colegiados de decisión técnico pedagógica de cada plantel educativo, los
cuales tendrán a su cargo adoptar e implementar las decisiones para
contribuir al máximo logro de aprendizaje de los educandos, el
desarrollo de su pensamiento crítico y el fortalecimiento de los lazos
entre escuela y comunidad

Por su parte, el artículo 110 señala que la educación tendrá un proceso
de mejora continua, el cual implica el desarrollo permanente del Sistema
Educativo Nacional para el incremento del logro académico de los
educandos. Tendrá como eje central el aprendizaje de niñas, niños,
adolescentes y jóvenes de todos los tipos, niveles y modalidades
educativos.

En concordancia con la Ley General, la Ley Reglamentaria del Artículo
3o. de la Constitución Política de los Estados Unidos Mexicanos, en
materia de Mejora Continua de la Educación (D.O.F. 30 de octubre de
2019) tiene como objeto regular el Sistema Nacional de Mejora Continua
de la Educación, así como el organismo que lo coordina, al que se
denominará Comisión Nacional para la Mejora Continua de la Educación y
el Sistema Integral de Formación, Actualización y Capacitación que será
retroalimentado por evaluaciones diagnósticas.

El Sistema Nacional de Mejora Continua de la Educación es un conjunto de
actores, instituciones y procesos estructurados y coordinados, que
contribuyen a la mejora continua de la educación, para dar cumplimiento
a los principios, fines y criterios establecidos en la Constitución
Política de los Estados Unidos Mexicanos, en la Ley General de Educación
y en la presente Ley (art. 4).

De acuerdo al artículo 6, los principios del Sistema son: I. El
aprendizaje de las niñas, niños, adolescentes y jóvenes, como centro de
la acción del Estado para lograr el desarrollo armónico de todas sus
capacidades orientadas a fortalecer su identidad como mexicanas y
mexicanos, responsables con sus semejantes y comprometidos con la
transformación de la sociedad de que forman parte; II. La mejora
continua de la educación que implica el desarrollo y fortalecimiento
permanente del Sistema Educativo Nacional para el incremento del logro
académico de los educandos; III. El reconocimiento de las maestras y los
maestros como agentes fundamentales del proceso educativo y de la
transformación social; IV. La búsqueda de la excelencia en la educación,
entendida como el mejoramiento integral constante que promueve el máximo
logro de aprendizaje de los educandos, para el desarrollo de su
pensamiento crítico y el fortalecimiento de los lazos entre escuela y
comunidad, considerando las capacidades, circunstancias, necesidades,
estilos y ritmos de aprendizaje de los educandos; V. La integralidad del
Sistema Educativo Nacional, procurando la continuidad, complementariedad
y articulación de la educación, desde el nivel inicial hasta el tipo
superior; VI. La contribución para garantizar una cobertura universal en
todos los tipos y niveles educativos, y VII. La participación social y
comunitaria. Todo lo anterior en concordancia con el enfoque de derechos
humanos, de igualdad sustantiva y de respeto irrestricto a la dignidad
de las personas, así como del carácter obligatorio, universal,
inclusivo, intercultural, integral, público, gratuito, de excelencia y
laico de la educación que imparte el Estado y la rectoría que éste
ejerce, de conformidad con los fines establecidos en el artículo 3o. de
la Constitución Política de los Estados Unidos Mexicanos, para lograr la
mejora continua de la educación.
\end{quote}

\hypertarget{acerca-de-la-nueva-escuela}{%
\subsection{Acerca de la nueva
escuela}\label{acerca-de-la-nueva-escuela}}

\begin{quote}
De acuerdo al artículo 11 de la Ley General de Educación, el Estado, a
través de la nueva escuela mexicana, buscará la equidad, la excelencia y
la mejora continua en la educación, para lo cual colocará al centro de
la acción pública el máximo logro de aprendizaje de las niñas, niños,
adolescentes y jóvenes. Tendrá como objetivos el desarrollo humano
integral del educando, reorientar el Sistema Educativo Nacional, incidir
en la cultura educativa mediante la corresponsabilidad e impulsar
transformaciones sociales dentro de la escuela y en la comunidad.

El artículo 12 señala que la prestación de los servicios educativos se
impulsará el desarrollo humano integral para: I. Contribuir a la
formación del pensamiento crítico, a la transformación y al crecimiento
solidario de la sociedad, enfatizando el trabajo en equipo y el
aprendizaje colaborativo; II. Propiciar un diálogo continuo entre las
humanidades, las artes, la ciencia, la tecnología y la innovación como
factores del bienestar y la transformación social; III. Fortalecer el
tejido social para evitar la corrupción, a través del fomento de la
honestidad y la integridad, además de proteger la naturaleza, impulsar
el desarrollo en lo social, ambiental, económico, así como favorecer la
generación de capacidades productivas y fomentar una justa distribución
del ingreso; IV. Combatir las causas de discriminación y violencia en
las diferentes regiones del país, especialmente la que se ejerce contra
la niñez y las mujeres, y V. Alentar la construcción de relaciones
sociales, económicas y culturales con base en el respeto de los derechos
humanos.
\end{quote}

\hypertarget{acerca-de-los-planes-de-estudio}{%
\subsection{Acerca de los planes de
estudio}\label{acerca-de-los-planes-de-estudio}}

\begin{quote}
Los planes de estudio deben favorecer el desarrollo integral y gradual
de los educandos en los niveles preescolar, primaria, secundaria, el
tipo media superior y la normal, considerando la diversidad de saberes,
con un carácter didáctico y curricular diferenciado, que responda a las
condiciones personales, sociales, culturales, económicas de los
estudiantes, docentes, planteles, comunidades y regiones del país. La
Secretaría de Educación Pública, de acuerdo al artículo 23 de la Ley
General de Educación, determinará los planes y programas de estudio,
aplicables y obligatorios en toda la República Mexicana, de la educación
preescolar, la primaria, la secundaria, la educación normal y demás
aplicables para la formación de maestras y maestros de educación básica,
de conformidad a los fines y criterios establecidos en los artículos 15
y 16 de esta Ley.

El proceso educativo que se genere a partir de la aplicación de los
planes y programas de estudio se basará en la libertad, creatividad y
responsabilidad que aseguren una armonía entre las relaciones de
educandos y docentes; a su vez, promoverá el trabajo colaborativo para
asegurar la comunicación y el diálogo entre los diversos actores de la
comunidad educativa. Los libros de texto que se utilicen para cumplir
con los planes y programas de estudio para impartir educación por el
Estado y que se derive de la aplicación del presente Capítulo, serán los
autorizados por la Secretaría en los términos de esta Ley, por lo que
queda prohibida cualquier distribución, promoción, difusión o
utilización de los que no cumplan con este requisito. Las autoridades
escolares, madres y padres de familia o tutores harán del conocimiento
de las autoridades educativas correspondientes cualquier situación
contraria a este precepto.

De acuerdo al artículo 29, en los planes de estudio se establecerán: I.
Los propósitos de formación general y, en su caso, la adquisición de
conocimientos, habilidades, capacidades y destrezas que correspondan a
cada nivel educativo; II. Los contenidos fundamentales de estudio,
organizados en asignaturas u otras unidades de aprendizaje que, como
mínimo, el educando deba acreditar para cumplir los propósitos de cada
nivel educativo y que atiendan a los fines y criterios referidos en los
artículos 15 y 16 de esta Ley; III. Las secuencias indispensables que
deben respetarse entre las asignaturas o unidades de aprendizaje que
constituyen un nivel educativo; IV. Los criterios y procedimientos de
evaluación y acreditación para verificar que el educando cumple los
propósitos de cada nivel educativo; V. Los contenidos a los que se
refiere el artículo 30 de esta Ley, de acuerdo con el tipo y nivel
educativo, y VI. Los elementos que permitan la orientación integral del
educando establecidos en el artículo 18 de este ordenamiento.
\end{quote}

\hypertarget{legislaciuxf3n-aplicable-a-ciberseguridad}{%
\subsection{Legislación aplicable a
ciberseguridad}\label{legislaciuxf3n-aplicable-a-ciberseguridad}}

El marco legal aplicable a la ciberseguridad ({``México''}, s.~f./):

\begin{itemize}
\tightlist
\item
  Constitución Política de los Estados Unidos Mexicanos;
\item
  Ley de Telecomunicaciones y Radiodifusión (FTBL);
\item
  Ley Federal de Protección de Datos Personales en poder de Particulares
  (Ley de Protección de Datos), sus reglamentos, recomendaciones,
  directrices y reglamentos similares sobre protección de datos;
\item
  Norma Federal de Transparencia y Acceso a la Información Pública;
\item
  Normas Generales como la Norma Oficial Mexicana con respecto a los
  requisitos que deben observarse al guardar mensajes de datos;
\item
  Ley de instrumentos negociables y operaciones de crédito;
\item
  Código Tributario Federal mexicano;
\item
  Ley de Instituciones de Crédito;
\item
  Circular Única para Bancos;
\item
  Ley de Propiedad Industrial;
\item
  Ley mexicana de derechos de autor;
\item
  Código Penal Federal;
\item
  Norma de seguridad nacional;
\item
  Ley del Trabajo;
\item
  Ley de la Policía Federal;
\item
  Plan Nacional de Desarrollo 2013-2018;
\item
  Estrategia Nacional de Ciberseguridad 2017;
\item
  Programa Nacional de Seguridad Pública 2014-2018; y
\item
  Programa Nacional de Seguridad 2014-2018.
\end{itemize}

Existen dos estándares mexicanos para la ciberseguridad, que son
obligatorias en México y están basados en los estándares ISO/IEC:

\begin{itemize}
\tightlist
\item
  NMX-I-27001-NYCE-2015 (ISO/IEC 270001:2013)
\item
  NMX-I-27002-NYCE-2015 (ISO/IEC 270002:2013)
\end{itemize}

El Código Penal establece varios delitos en materia de ciberseguridad:

\begin{itemize}
\tightlist
\item
  Piratería
\item
  Phishing
\item
  Infección con malware
\item
  Posesión o uso de herramientas para cometer delitos informáticos
\item
  Robo de identidad (Art. 211 bis)
\end{itemize}

\hypertarget{legislaciuxf3n-aplicable-al-desarrollo-de-software}{%
\subsection{Legislación aplicable al desarrollo de
software}\label{legislaciuxf3n-aplicable-al-desarrollo-de-software}}

Las principales consideraciones legales que aplican de manera general al
desarrollo de software (\emph{ASPECTOS LEGALES RELACIONADOS CON EL
DESARROLLO Y USO DEL SOFTWARE}, s.~f.):

\begin{itemize}
\item
  Constitución Política de los Estados Unidos Mexicanos

  \begin{itemize}
  \tightlist
  \item
    El derecho de acceso a las tecnologías de la información se
    establece en el Art. 6.
  \end{itemize}
\item
  Ley de Propiedad Industrial;
\item
  Ley Federal del Derecho de Autor;
\item
  Código Federal de Procedimientos Civiles;
\item
  Ley Federal de Procedimiento Administrativo;
\item
  Código Penal Federal;
\end{itemize}

Dependiendo de la industria para la cual se esté desarrollando aplicará
la normatividad de esa industria.

\hypertarget{estado-del-arte}{%
\section{Estado del arte}\label{estado-del-arte}}

\hypertarget{paradigma-pedaguxf3gico-para-la-actividad}{%
\subsection{Paradigma pedagógico para la
actividad}\label{paradigma-pedaguxf3gico-para-la-actividad}}

Desde la perspectiva del constructivismo la programación es una
actividad compleja basada en operaciones proposicionales (Piaget et~al.,
2016).

El desarrollo de las operaciones proposicionales comparte
características con el desarrollo del lenguaje y se puede utilizar
tecnología para facilitar ese desarrollo por medio de actividades que
desarrollan progresivamente el conocimiento (s.~f.).

Como la programación comparte muchos atributos con la adquisición del
lenguaje, revisaremos el trabajo de Bruner (Snow, 1979).

El aprendizaje se puede estimular usando elementos emocionales (s.~f.).

Las actividades que se desarrollen para este sistema deben proveer
pistas a los estudiantes para generar las habilidades que están dentro
de su zona de conocimiento proximal (s.~f.).

\hypertarget{tecnologuxeda-para-implementar-la-evaluaciuxf3n-automuxe1ticamente}{%
\subsection{Tecnología para implementar la evaluación
automáticamente}\label{tecnologuxeda-para-implementar-la-evaluaciuxf3n-automuxe1ticamente}}

\href{https://people.eecs.berkeley.edu/~necula/cil/}{CIL -
Infrastructure for C Program Analysis and Transformation (v. 1.3.7)}.

\href{https://www.researchgate.net/post/Is_there_any_tool_parser_to_extract_information_from_C_code}{Las
recomendaciones de Research Gate}.

\href{https://bytes.com/topic/c/answers/219149-extract-function-c-code}{Podrìa
usarse \texttt{cscope}}.

\href{https://cboard.cprogramming.com/c-programming/97667-how-extract-defined-function-names-c-file.html}{Un
formo mostrando cómo extraer funciones de programas en C}.

\href{https://www.codingalpha.com/lexical-analyzer-in-c-programming/}{Analizador
Léxico de C}.

\href{https://github.com/lotabout/write-a-C-interpreter/blob/master/tutorial/en/3-Lexer.md}{Una
introducción tomada de un libro donde se construye un intérprete de C}.

\href{https://medium.com/young-coder/how-i-wrote-a-lexer-39f4f79d2980}{Un
artículo acerca de cómo alguien construyó un analizador léxico}.

\href{https://www.programmingassignmenthelper.com/lexer-and-parser-in-c/}{Construir
un analizador léxico y un intérprete en C}.

\href{https://www.thecrazyprogrammer.com/2017/02/lexical-analyzer-in-c.html}{Cómo
construir un analizador léxico en C y C++}.

\href{https://www.lysator.liu.se/c/ANSI-C-grammar-l.html}{Especificación
del Analizador léxico usando yacc para el estándar de ANSI C}.

\href{https://codemirror.net/6/}{Un editor de código escrito en JS que
usa un Árbol de Interpretación Completo}.

\href{https://github.com/google/sanitizers/wiki/AddressSanitizer}{\texttt{AddressSanitizer}
permite probar errores en memoria y es parte de LLVM y GCC} y
probablemente es la opción usada en el \texttt{makefile} de Exercism.
\href{https://doc.coreboot.org/technotes/asan.html}{El manual de
\texttt{coreboot} tiene un pequeño tutorial para usarlo}.

\hypertarget{hipuxf3tesis}{%
\section{Hipótesis}\label{hipuxf3tesis}}

\begin{enumerate}
\def\labelenumi{\arabic{enumi}.}
\item
  Para desarrollar habilidades de seguridad es mejor la
  retroalimentación inmediata y continua es mejor que la
  retroalimentación únicamente al tiempo de evaluación.

  \textbf{Hipótesis Nula}: No hay diferencia en el desarrollo de
  habilidades de ciberseguridad entre usar retroalimentación inmediata
  automatizada y la retroalimentación usada por el profesor.
\item
  Los estudiantes que usan la revisión automatizada preferirían seguir
  utilizando revisión automatizada de su código.

  \textbf{Hipótesis Nula}: Los estudiantes serían indiferentes o
  rechazarían el uso de la revisión automatizada.
\end{enumerate}

\hypertarget{delimitaciuxf3n-del-tema}{%
\section{Delimitación del tema}\label{delimitaciuxf3n-del-tema}}

El alcance de esta investigación corresponde al espacio temporal de un
año, comenzando en enero de 2021 y terminando en diciembre de 2021.

Está acotado a los resultados que pueden obtenerse en la evaluación de
los estudiantes de la Universidad Iberoamericana en las asignaturas de
\emph{Seguridad e Integridad de la Información y Laboratorio}, y
acotadas por el acceso a los resultados de las evaluaciones.

Por la limitante del tiempo y a pesar de que está entre mis objetivos de
vida desarrollar una buena secuencia didáctica para guiar el desarrollo
de conceptos y abstracciones necesarias para programar efectivamente,
este trabajo se limitará a evaluar el efecto de usar un régimen de
retroalimentación continua usando sistema automático de evaluación de
estilo de código en el rendimiento de los estudiantes.

De esta manera los objetivos son específicos, conmensurables,
alcanzables, dependen únicamente de mi responsabilidad, y están acotados
temporalmente.

De esta manera, este trabajo plantea las bases para medir el desempeño
de la secuencia didáctica que se desarrollará en etapas posteriores.

\hypertarget{preguntas-de-investigaciuxf3n}{%
\section{Preguntas de
investigación}\label{preguntas-de-investigaciuxf3n}}

\begin{enumerate}
\def\labelenumi{\arabic{enumi}.}
\tightlist
\item
  ¿Cuáles son los mecanismos más usados actualmente para enseñar la
  programación?
\item
  ¿Qué tan efectivas son esas metodologías en la enseñanza de
  programación?
\item
  ¿Qué elementos contienen estas metodologías?
\item
  ¿Cómo pueden aprovecharse efectivamente estos elementos?
\item
  ¿Qué conceptos deben aprender los alumnos para programar
  efectivamente?
\item
  ¿Cómo deben organizarse esos conceptos para que el proceso educativo
  sea eficiente?
\item
  ¿Cómo puede evaluarse la adquisición de esos conceptos?
\item
  ¿Qué mecanismos de evaluación se usan actualmente?
\item
  ¿Qué tan efectivos son esos mecanismos para evaluar la adquisición de
  habilidad de los estudiantes?
\item
  ¿Cuáles son los hábitos de los mejores programadores?
\item
  ¿Cómo se pueden fomentar esos hábitos en el salón de clase?
\item
  ¿Cómo la evaluación puede ayudar a fomentar esos hábitos?
\item
  ¿Qué tanto es el efecto de la evaluación en la formación de esos
  hábitos?
\item
  ¿Cuál es el efecto de conocer el detalle de la evaluación en la
  habilidad de los estudiantes?
\item
  ¿Cuál es el efecto de recordar el detalle de la evaluación en la
  habilidad de los estudiantes?
\item
  Si se recordó constantemente los objetivos de evaluación y eso mostró
  mejora en los resultados de la evaluación, ¿Qué tanto es efecto de los
  hábitos y qué tanto es el efecto de «entrenar para el examen»?
\end{enumerate}

\newpage

\hypertarget{objetivo-general}{%
\section{Objetivo General}\label{objetivo-general}}

Proponer un sistema automatizado para la enseñanza de ciberseguridad
fundamentado en el paradigma constructivista.

\hypertarget{objetivos-especuxedficos}{%
\section{Objetivos específicos}\label{objetivos-especuxedficos}}

\begin{enumerate}
\def\labelenumi{\arabic{enumi}.}
\item
  Definir criterios para evaluar un trabajo de ciberseguridad.
\item
  Diseñar evaluación de acuerdo a los criterios definidos.
\item
  Implementar sistema automatizado de evaluación.
\end{enumerate}

\hypertarget{esquema}{%
\section{Esquema}\label{esquema}}

Como fundamento del desarrollo del sistema completo:

\begin{itemize}
\tightlist
\item
  Desarrollo del pensamiento abstracto de acuerdo al constructivismo de
  Piaget.
\item
  Adquisición del lenguaje de Brunner.
\item
  Teoría del horizonte proximal del conocimiento Vigotsky.
\item
  Aprendizaje por descubrimiento.
\end{itemize}

En términos de la disciplina de programación:

\begin{itemize}
\tightlist
\item
  Buenas prácticas de programación.
\item
  Diferencia entre buenos programadores.
\item
  Integración continua.
\item
  Desarrollo Basado en Pruebas.
\item
  Mantenibilidad del código.
\end{itemize}

En términos de la enseñanza de ciberseguridad:

\begin{itemize}
\tightlist
\item
  Prácticas comunes en la enseñanza de programación.
\item
  Prácticas efectivas en la enseñanza de educación.
\end{itemize}

Para poder diseñar y procesar los datos del experimento:

\begin{itemize}
\tightlist
\item
  Diseño experimental.
\item
  Poder estadístico.
\item
  Tamaño de efecto.
\end{itemize}

\hypertarget{diseuxf1ando-los-juegos-para-el-aprendizaje}{%
\subsection{Diseñando los juegos para el
aprendizaje}\label{diseuxf1ando-los-juegos-para-el-aprendizaje}}

Existen plataformas de juegos como \href{https://overthewire.org/}{Over
The Wire} (p.e. \href{https://overthewire.org/wargames/bandit/}{bandit})
que plantean retos para que aprendas haciendo y ofrecen pistas para usar
la computadora más allá del \emph{point and click}.

Derivado de la expleriencia al usar los CTF disponibles en el
\href{/csaw/}{taller de ciberseguridad} mis alumnos han tenido problemas
porque:

\begin{itemize}
\item
  En algunos casos la dificultad salta de un problema súper sencillo a
  el nivel más alto de exploración del problema.

  Para los estudiantes es frustrante pasar varias sesiones explorando
  sin avance aparente.
\item
  Los juegos están completamente disociados de los conocimiento teóricos
  que se requieren.

  Esto puede ser una ventaja, porque dependiendo de su uso podemos
  entrenar a los estudiantes en el uso de habilidades metacognitivas,
  como la resolución de problema, la identificación de incógnitas y la
  investigación autodidacta.
\end{itemize}

En el presente trabajo se plantea
\href{/techno/aplicaciones-vulnerables-para-poc/}{generar un camino de
victorias intermedias} para aprender los conceptos desde una perspectiva
práctica, con ayudas en el camino.

De igual manera, se trabaja una metodología para enseñar habilidades
metacognitivas.

\begin{center}\rule{0.5\linewidth}{0.5pt}\end{center}

Para el diseño de las actividades de aprendizaje en ciberseguridad 2022,
se utilizó una estrategia múltiple:

\begin{itemize}
\item
  Diseñar actividades dirigidas

  \begin{itemize}
  \item
    Injección de código.
  \item
    Desbordamiento de memoria.
  \item
    Formato de cadenas.
  \end{itemize}
\item
  Proyecto de generación de aplicaciones y remediación de seguridad

  Este proyecto puede tomarse como base para actividades futuras.
\end{itemize}

\hypertarget{desarrollo-de-habilidades-meta-cognitivas}{%
\subsection{Desarrollo de habilidades
meta-cognitivas}\label{desarrollo-de-habilidades-meta-cognitivas}}

Una característica de la ciberseguridad es que el área de trabajo está
enfocada en lo que no se conoce, porque en cuanto los problemas de
seguridad se vuelven conocidos, existe un esfuerzo de la industria por
mitigarlos y corregirlos.

En una aproximación a la aplicación de los juegos para aprendizaje con
estudiantes se detectó que si se exponía la solución a los estudiantes,
los estudiantes intentan aprenderla como procedimiento, pero no aprenden
una metodología para buscar vulnerabilidades en general.

\hypertarget{selecciuxf3n-de-criterios-candidatos-a-evaluaciuxf3n-automuxe1tica}{%
\section{Selección de criterios candidatos a evaluación
automática}\label{selecciuxf3n-de-criterios-candidatos-a-evaluaciuxf3n-automuxe1tica}}

Esta lista de errores comunes puede revisarse a partir de una respuesta
correcta con una prueba unitaria.

Estos errores no están relacionados con el estilo de código, pero por su
frecuencia y simplicidad de implementación de la automatización,
permiten implementarse de manera sencilla.

\begin{itemize}
\tightlist
\item
  Tipo de variable incorrecto
\item
  Falta de la sentencia para devolver la variable.
\item
  Error de desbordamiento de buffer
\item
  Error por uno
\item
  La variable está llamada por valor y no por referencia
\end{itemize}

Categorización de criterios criterios:

1). Se puede revisar automáticamente.

2). No se puede revisar automáticamente.

3). Debería ser posible, pero se necesita más investigación.

\hypertarget{proyecto-a-futuro}{%
\subsection{Proyecto a futuro}\label{proyecto-a-futuro}}

El alumno presenta en su proyecto los siguientes elementos:\footnote{No
  se puede revisar automáticamente.}

\begin{itemize}
\item
  Levantamiento de requerimiento (8 pasos)
\item
  Diagrama de entradas, procesos y salidas
\item
  Diagrama de flujo (opcional)
\item
  Pseudocódigo (repositorio de git)
\item
  Pruebas de escritorio
\end{itemize}

El programa usa el estilo K\&R.\footnote{Se puede revisar
  automáticamente.}

\begin{itemize}
\tightlist
\item
  El programa indent puede corregir el estilo pero se puede un script
  que únicamente revise.
\end{itemize}

El alumno puede identificar y corregir código donde las únicas fallas
sean:\footnote{Se puede revisar automáticamente.}

\begin{itemize}
\tightlist
\item
  Tipo de variable incorrecto.
\item
  Falta de la sentencia para devolver la variable.
\item
  Error de desbordamiento de buffer.
\item
  Error por uno.
\item
  La variable está llamada por valor y no por referencia.
\end{itemize}

Identificación de variables (sustantivos).\footnote{No se puede revisar
  automáticamente.}

\begin{itemize}
\tightlist
\item
  Las variables usadas en el programa utilizan el lenguaje de la
  declaración del problema. (Un diccionario del dominio del problema)
\item
  Los nombres de las variables no dan información acerca de la
  implementación. (Un diccionario de palabras del lenguaje)
\end{itemize}

Identificación de funciones (verbos).\footnote{No se puede revisar
  automáticamente.}

\begin{itemize}
\item
  Las funciones usadas en el programa utilizan el lenguaje de la
  declaración del problema. (Un diccionario del dominio del problema)
\item
  Los nombres de las funciones no dan información acerca de la
  implementación. (Un diccionario de palabras del lenguaje)
\item
  Declaración de constantes (mayúsculas).
\item
  Todos los DEFINE y const van en mayúsculas.\footnote{Debería ser
    posible, pero se necesita más investigación.}
\item
  Identificar el estilo de declaración de variables, constantes y
  funciones.
\end{itemize}

\hypertarget{estuxe1ndar-para-la-organizaciuxf3n-del-proyecto}{%
\subsection{Estándar para la organización del
proyecto}\label{estuxe1ndar-para-la-organizaciuxf3n-del-proyecto}}

La estructura de los proyectos sigue el siguiente estándar:\footnote{Debería
  ser posible, pero se necesita más investigación.}

\begin{itemize}
\tightlist
\item
  Cada función se declara en un archivo diferente ``.c''. Existe al
  menos un directorio para alojar las funciones del programa. Se genera
  un archivo ``.h'' para declarar las funciones de cada directorio. El
  makefile declara el directorio para vincular la biblioteca generada.
  El makefile compila el proyecto correctamente con gcc.
\end{itemize}

Ventajas

\begin{itemize}
\item
  Se genera de manera orgánica bibliotecas reusable.
\item
  Esta organización permite incluir objetos individualmente según se
  usen. (p.e. cargar la biblioteca para usar una única función no
  implica vincular los diez objetos que conforman la biblioteca).
\end{itemize}

Desventajas

\begin{itemize}
\item
  Hay muchos archivos por cada programa.
\item
  Pueden estar desorganizados.
\item
  La categorización de las bibliotecas es arbitraria y no puede
  evaluarse automáticamente
\item
  Puede generar frustración y recelo por C o programar por el sobre
  trabajo que estos requisitos solicitan.
\item
  No deben existir definiciones de variables globales.
\item
  Se genera a partir del código la documentación del código al pasar por
  doxygen. La documentación describe adecuadamente el código (no se
  puede evaluar automáticamente). Podrían usarse test cases en doxygen
  para evaluar esto, si doxygen lo admite.
\end{itemize}

\hypertarget{registros-en-git}{%
\subsection{Registros en Git}\label{registros-en-git}}

Esta es un área muy importante para garantizar que el código es
mantenible y para permitir una adecuada revisión de los cambios.

Desafortunadamente, es un área que muy posiblemente sería imposible de
evaluar de manera automática, salvo tal vez con la aplicación de
algoritmos de aprendizaje de máquina, y aún así las conclusiones
deberían ser supervisadas por una persona para asegurarse de la calidad
de la revisión.

Un buen historial de git:\footnote{Se puede revisar automáticamente.}

\begin{itemize}
\tightlist
\item
  Tiene un commit por cada avance. (Un proxy para esto sería un número
  aproximado de commits dependiendo de la magnitud del proyecto).
\item
  Describe lo que se hizo en el commit.
\item
  Explica por qué se hacen los cambios si es necesario.
\item
  Tiene pocas líneas para revisar (se puede usar un número arbitrario de
  líneas agregadas, sin límite para eliminadas).
\end{itemize}

La metodología para el diseño de los retos se utilizó para el
\href{https://preuniversitarios.ibero.mx/CSAW/}{Cyber Security Challenge
for High School}.

Se encontró que implementación técnica de los juegos puede afectar el
objetivo de aprendizaje:

\begin{itemize}
\tightlist
\item
  Pueden existir atajos para resolver los juegos.
\end{itemize}

\hypertarget{lecciones-aprendidas-del-primer-capture-the-flag-que-hice}{%
\subsection{\texorpdfstring{Lecciones aprendidas del primer
\emph{Capture the Flag} que
hice}{Lecciones aprendidas del primer Capture the Flag que hice}}\label{lecciones-aprendidas-del-primer-capture-the-flag-que-hice}}

\begin{enumerate}
\def\labelenumi{\arabic{enumi}.}
\item
  Si se va a publicar el binario para permitir pruebas locales, es muy
  \textbf{importante separar las banderas del binario}, o cambiarlas en
  el binario publicado.

  No hacer esto puede eliminar el aprendizaje en el juego, porque
  cualquiera abre un block de notas o usa \texttt{strings}.

  De igual manera, leí que se recomienda no usar un indicativo que haga
  evidente las banderas como \texttt{\textbackslash{}flag\{\}} porque la
  gente podría usar \texttt{grep}.
\item
  Para simplificar los retos conviene
  \href{./aplicaciones-vulnerables-para-poc/}{desactivar las opciones de
  seguridad del compilador} y eliminar las optimizaciones del código.

\begin{verbatim}
CFLAGS=-fno-stack-protector -no-pie -ggdb3 -O0
\end{verbatim}
\item
  Al intentar explotar \href{}{un binario con protecciones
  desactivadas}, se puede quedar el programa colgado y utilizando el
  máximo de los recursos. Por lo tanto conviene limitar la ejecución del
  código.

\begin{verbatim}
prlimit -t=${TIEMPO_RAZONABLE_PARA_EJECUTAR} aplicación
\end{verbatim}

  (Tengo entendido que este es tiempo efectivo de ejecución y no hay
  problema si el programa está esperando input; no he puesto a prueba
  esta hipótesis.)
\item
  Las aplicaciones usan un buffer para imprimir en \texttt{stdout}. Si
  la aplicación no limpia el buffer al enviar la bandera, puede que el
  jugador resuelva correctamente pero no obtenga la bandera.
\end{enumerate}

\hypertarget{diseuxf1ando-los-juegos-para-el-aprendizaje-1}{%
\subsection{Diseñando los juegos para el
aprendizaje}\label{diseuxf1ando-los-juegos-para-el-aprendizaje-1}}

Algo que me encanta de plataformas de juegos como {[}Over The
Wire{]}(https://overthewire.org/ ``The wargames OverTheWire can help you
to learn and practice security concepts in the form of fun-filled
games.'' (puedes empezar por
\href{https://overthewire.org/wargames/bandit/}{bandit}) es que plantea
retos para que aprendas haciendo y te ofrece las pistas suficientes para
que vayas aprendiendo a usar tu computadora más allá del \emph{point and
click}.

En el caso del segundo juego «natas», cuando lo intenté con el
\href{/csaw/}{taller de ciberseguridad} mis alumnos han tenido problemas
porque en algunos casos la dificultad salta de un problema súper
sencillo a el nivel más alto de exploración del problema.

Y aunque es divertido para un hacker que ya tiene algo de experiencia y
al menos tiene una noción de qué hay que hacer, es frustrante pasar
sesione explorando sin avance aparente.

Entonces he querido
\href{/techno/aplicaciones-vulnerables-para-poc/}{hacer un camino de
victorias intermedias} para aprender los conceptos desde una perspectiva
práctica, \href{/techno/aplicacines-vulnerables-para-poc/}{con ayudas en
el camino}.

\hypertarget{desactivar-medidas-de-seguridad}{%
\subsection{Desactivar medidas de
seguridad}\label{desactivar-medidas-de-seguridad}}

La mayoría de las vulnerabilidades en los programas son errores de
gestión de memoria.

Para enseñar cómo funcionan los ataques de desbordamiento de la pila, es
importante entender
\href{https://www.geeksforgeeks.org/memory-layout-of-c-program/}{cómo
guardan información los programas}.

\href{https://www.eecs.umich.edu/courses/eecs588/static/stack_smashing.pdf}{«Smashing
the stack for fun and profit»} explica relativamente bien cómo funciona
todo.

\href{https://web.archive.org/web/20060919233718/https://inst.eecs.berkeley.edu/~cs164/sp05/ia32-refs/ia32-chapter-three.pdf}{Cómo
funciona la pila en ia32}..

\href{https://eli.thegreenplace.net/2011/09/06/stack-frame-layout-on-x86-64}{Cómo
funciona la pila en x86\_64}.

\href{https://courses.cs.washington.edu/courses/cse351/18wi/lectures/10/CSE351-L10-x86-III_18wi.pdf}{Más
acerca de como funciona el stack en x86-64}.

\hypertarget{cuxf3mo-se-protegen-los-programas}{%
\subsection{Cómo se protegen los
programas}\label{cuxf3mo-se-protegen-los-programas}}

Al compilar los programas se pueden activar varias opciones de
protección.

\href{https://mcuoneclipse.com/2019/09/28/stack-canaries-with-gcc-checking-for-stack-overflow-at-runtime/}{\texttt{-stack-protection}}:
verifica que las variables se escriban en su lugar.

\texttt{-pie}: genera código que puede moverse en la memoria.

\texttt{-s}: elimina (strips) los símbolos de depuración.

\texttt{-\/-\/-noexecstack}: marca la memoria escribible como no
ejecutable (DEP)

Se pueden revisar opciones de seguridad de los binarios usando:

\begin{verbatim}
checksec --file=${EJECUTABLE}
checksec --directory=${EJECUTABLE}
\end{verbatim}

\href{https://wiki.debian.org/Hardening}{Guía de Debian para proteger
los programas}.

\hypertarget{cuxf3mo-demostrar-programas-vulnerables}{%
\subsection{Cómo demostrar programas
vulnerables}\label{cuxf3mo-demostrar-programas-vulnerables}}

Para aprender los básicos acerca de cómo atacar programas, conviene
\href{https://stackoverflow.com/questions/2340259/how-to-turn-off-gcc-compiler-optimization-to-enable-buffer-overflow}{deshabilitar
las funciones de seguridad en la compilación de los programas},
\href{https://reverseengineering.stackexchange.com/a/27682}{incluyendo
RELRO}.

\begin{verbatim}
gcc \
    -fno-stack-protector \
    -no-pie \
    -Wl,-z,norelro,execstack \
    -o ${PROGRAM} \
    ${SOURCE}
\end{verbatim}

En varios sitios recomiendan además
\href{https://0x10f8.wordpress.com/2019/05/21/subverting-nx-bit-with-return-to-libc/}{desactivar
ASLR en la configuración del kernel}:

\begin{verbatim}
# echo 0 > /proc/sys/kernel/randomize_va_space
\end{verbatim}

Sobrescribir la pila, permite controlar el flujo de ejecución del
programa porque \href{http://theamazingking.com/tut2.php}{en la pila se
guardan los punteros para continuar la ejecución del programa}.

\hypertarget{ejecutar-programas-desde-la-pila}{%
\subsection{Ejecutar programas desde la
pila}\label{ejecutar-programas-desde-la-pila}}

En la mayoría de los casos los sistemas operativos actualizados impiden
la ejecución de código en la pila aunque
\href{https://stackoverflow.com/a/3756315/10467443}{es posible que
algunos dispositivos embebidos aún contengan esta vulnerabilidad}

\href{https://stackoverflow.com/a/3755573/10467443}{Esta respuesta de
Stack Overflow} lista algunos punteros para desactivar las protecciones:

\begin{quote}
``- Disable stack protection at the OS level. - Allow execution from the
stack on a particular executable file. - mprotect() the stack. - Maybe
some other things\ldots{}''
\end{quote}

Esta medida de seguridad es difícil de desactivar porque
\href{https://security.stackexchange.com/a/230816/274242}{se tiene que
desactivar en todo el sistema operativo y no puede desactivarse
únicamente para un programa o dentro de un contenedor}.

Por lo tanto, para montar estos juegos, estoy considerando usar alguna
máquina virtual ultra ligera:

\begin{itemize}
\item
  Alpine sobre \texttt{QEMU}:

  Sospecho que esta va a ser la forma más sencilla de levantar un
  servicio que tenga desactivado ASLR.
\item
  \href{https://github.com/miklevin/Levinux}{Levinux}:

  Me llamó la atención porque funciona en los 3 sistemas operativos
  mayoritarios tiene un perspectiva educativa y al parecer algunos
  tutoriales.
  \href{https://mikelev.in/2011/08/virtual-machine-runs-anywhere/}{Aquí
  está un post acerca de cómo se hizo Levinux}. Al final es un
  \texttt{QEMU}.
\end{itemize}

\hypertarget{cuxf3mo-aprovechar-la-vulnerabilidad-de-los-programas-para-ejecutar-cuxf3digo-arbitrario}{%
\subsection{Cómo aprovechar la vulnerabilidad de los programas para
ejecutar código
arbitrario}\label{cuxf3mo-aprovechar-la-vulnerabilidad-de-los-programas-para-ejecutar-cuxf3digo-arbitrario}}

Si tienes suficiente espacio en un programa vulnerable, puedes escribir
una función arbitraria en la memoria del programa.

Hay \href{http://shell-storm.org/shellcode/}{inventarios de funciones en
ensamblador para ataques de overflow}

En caso contrario,
\href{https://insecure.org/stf/smashstack.html}{puedes escribir el
código en el entorno y apuntar el puntero de retorno a tu código}.

\hypertarget{cuxf3mo-atacar-la-memoria-cuuxe1ndo-estuxe1-protegida}{%
\subsection{Cómo atacar la memoria cuándo está
protegida}\label{cuxf3mo-atacar-la-memoria-cuuxe1ndo-estuxe1-protegida}}

Se puede utilizar el código que ya está en memoria para construir
programas arbitrarios, eso se conoce como
\href{https://en.wikipedia.org/wiki/Return-oriented_programming}{return-oriented
programming (ROP)} y
\href{https://seclists.org/bugtraq/1997/Aug/63}{este es uno de los
primeros ejemplos}.

\href{http://shell-storm.org/talks/ROP_course_lecture_jonathan_salwan_2014.pdf}{Se
pueden generar ataques elaborados con las instrucciones en memoria}.

\href{https://github.com/JonathanSalwan/ROPgadget/tree/master}{Esta
herramienta busca código disponible para construir tus programas}.

\hypertarget{cuxf3mo-evitar-que-se-diagnostiquen-los-programas}{%
\subsection{Cómo evitar que se diagnostiquen los
programas}\label{cuxf3mo-evitar-que-se-diagnostiquen-los-programas}}

\href{http://shell-storm.org/blog/Linux-process-execution-and-the-useless-ELF-header-fields/}{Eliminar
los campos de ELF que no se usan para ejecutar impide su depuración} y
\href{http://shell-storm.org/blog/Linux-process-execution-and-the-useless-ELF-header-fields/anti-debug.c}{aquí
hay código para hacerlo}.

También
\href{http://shell-storm.org/blog/Polymorphism-and-Return-Oriented-Programming/}{se
puede alterar un programa por otro funcionalmente equivalente que no
contenga la parte común de los exploits}.

\href{https://stackoverflow.com/questions/5752605/tool-to-clearly-visualize-memory-layout-of-a-c-program}{Herramientas
para visualizar la memoria}:

\begin{itemize}
\tightlist
\item
  \href{https://www.cs.kent.ac.uk/projects/gc/gcspy/}{GCspy}
\item
  \href{http://www.cs.kent.edu/~jmaletic/cs69995-PC/papers/Moreta07.pdf}{Paper
  de visualización donde no se muestra ni se hace referencia al código}
\end{itemize}

\hypertarget{herramientas-para-programar-con-cuxf3digo-existente}{%
\subsection{Herramientas para programar con código
existente}\label{herramientas-para-programar-con-cuxf3digo-existente}}

\href{https://triton.quarkslab.com/}{Análisis dinámico de binarios}

http://shell-storm.org/project/ROPgadget/

La idea es usar código que contiene funciones que retornan, porque se
pueden encadenar para alcanzar cualquier objetivo.

Esto se conoce como \textbf{Return Oriented Programming}. Tiene las
ventajas de que se salta todas las protecciones conocidas de la pila y
puede mitigarse \href{https://lwn.net/Articles/870045/}{borrando los
registros antes de retornar},
\href{https://www.jerkeby.se/newsletter/posts/rop-reduction-zero-call-user-regs/}{introduciendo
13\% JOP gadgets y reduciendo 36\% ROP gadgets para Linux}.

Sin embargo,
\href{https://dustri.org/b/paper-notes-clean-the-scratch-registers-a-way-to-mitigate-return-oriented-programming-attacks.html}{borrar
los registros podría deteriorar el rendimiento} presumiblemente sin
agregar a la seguridad.

https://www.jerkeby.se/newsletter/posts/history-of-rop/

\href{https://stackoverflow.com/questions/1345670/stack-smashing-detected\#comment37057047_1347464}{Un
consejo terrible: desactivar la protección anti pila en producción}.

\hypertarget{minimizar-el-tamauxf1o-de-los-ctf}{%
\subsection{Minimizar el tamaño de los
CTF}\label{minimizar-el-tamauxf1o-de-los-ctf}}

Las imágenes que \texttt{ctfcli} genera por defecto son del orden de
cientos de megabytes.

Aunque el espacio en disco es actualmente lo más barato en cómputo y en
varios lugares es gratuito tener varios GB de datos.

Sin embargo, para probar pequeños cambios en los juegos en un servidor,
tener que enviar cientos de MB por una pequeña diferencia no hace
sentido y toma mucho más tiempo hacer la retroalimentación y el
diagnóstico.

Así que pensé en hacer imágenes mínimas que contengan únicamente lo
necesario para ejecutar el reto. En algunos casos, sólo se necesita en
el contenedor los programas vulnerables compilados estáticamente y
\href{https://skarnet.org/software/s6}{la suite de supervisión}.

Una manera sencilla de hacer esto es compilar los programas y luego
generar un contenedor vacío que incluye:

\begin{itemize}
\tightlist
\item
  los programa necesarios
\item
  las bibliotecas que utilizan
\end{itemize}

El \texttt{Dockerfile} quedaría parecido a:

\begin{verbatim}
FROM debian:stretch-20211011-slim as builder

WORKDIR /root

RUN apt-get update \
  && apt-get -y install \
    build-essential \
  && apt clean

COPY src/makefile src/game.c /root/
RUN make game

FROM alpine:3.14.2 as system

RUN apk add \
        s6-networking \
        util-linux

WORKDIR /opt

COPY --from=builder /root/game /opt/app

FROM scratch

COPY --from=system \
  /lib/ld-musl-x86_64.so.1 \
  /lib/libskarnet.so.2.10 \
  /lib/libsmartcols.so.1 \
/lib/

COPY --from=system \
  /bin/s6-applyuidgid \
  /opt/app \
  /usr/bin/s6-tcpserver \
  /usr/bin/s6-tcpserver4d \
  /usr/bin/s6-tcpserver4-socketbinder \
  /usr/bin/prlimit \
/bin/

COPY --from=builder src/game /bin/app

EXPOSE 1337

CMD ["/bin/s6-tcpserver", "-u", "1001", "-g", "1001", "0.0.0.0", "1337", "/bin/prlimit", "-t=3", "/bin/app"]
\end{verbatim}

En algunas ocasiones, el ejercicio requiere que sea posible ejecutar una
consola, por lo que podría convenir incluir en el contenedor
\href{https://github.com/robxu9/bash-static/}{una versión estática de
bash} como \texttt{/bin/sh}.

Se está trabajando además en una máquina virtual mínima, y un sistema
para reproducirlo.

\hypertarget{implementaciuxf3n-de-infraestructura-para-proyecto}{%
\section{Implementación de infraestructura para
proyecto}\label{implementaciuxf3n-de-infraestructura-para-proyecto}}

\hypertarget{restringir-permisos-de-usuario-compartido-en-servidor}{%
\subsection{Restringir permisos de usuario compartido en
servidor}\label{restringir-permisos-de-usuario-compartido-en-servidor}}

Esta configuración se está llevando a cabo en el directorio
\$\{HOME\}/src/gitlab.com/gcd-ibero/gestion-de-infraestructura y se
puede acceder a él con el comando \texttt{cd\ \$(repo\ -inf)}.

\begin{itemize}
\item
  {[}✓{]} Utilizar un usuario compartido en un servidor para que los
  estudiantes usen su propia compu para ejecutar los comandos.

  \begin{itemize}
  \tightlist
  \item
    {[}✓{]} Generar el usuario.
  \item
    {[}✓{]} Instalar \texttt{byobu}.
  \item
    {[}✓{]} Generar una contraseña para el usuario
  \end{itemize}
\item
  {[}✓{]} Restringir los comandos que puede utilizar el usuario
  compartido:

  \href{https://www.ibm.com/support/pages/how-use-restricted-shell}{Cómo
  usar} una
  \href{https://www.gnu.org/software/bash/manual/html_node/The-Restricted-Shell.html}{consola
  restringida}

  \begin{itemize}
  \item
    {[}✓{]} Asignar \texttt{rbash} como consola.
  \item
    {[}✓{]} Generar un perfil que no puede modificarse y contiene las
    variables:

    \begin{itemize}
    \tightlist
    \item
      {[}✓{]} \texttt{PATH}
    \end{itemize}
  \item
    {[}✓{]} Agregar al directorio configurado en el \texttt{PATH} los
    comandos:

    \begin{itemize}
    \tightlist
    \item
      {[}✓{]} \texttt{ssh} para entrar al servidor de juegos
    \item
      {[}✓{]} \texttt{man} para consultar documentación
    \item
      {[}✓{]} \texttt{apropos} para consultar documentación
    \item
      {[}✓{]} \texttt{info} para consultar documentación
    \item
      {[}✓{]} \texttt{byobu} para compartir la pantalla
    \item
      {[}✓{]} \texttt{less}
    \item
      {[}✓{]} \texttt{nano}
    \item
      {[}✓{]} \texttt{clear}
    \item
      {[}✓{]} \texttt{tput}
    \end{itemize}
  \item
    {[}✓{]}
    \href{https://bryan-murdock.blogspot.com/2010/10/how-to-disable-ubuntu-command-not-found.html}{Desactivar
    el aviso de commando no encontrado}

\begin{verbatim}
unset command_not_found_handle
\end{verbatim}
  \end{itemize}
\end{itemize}

\hypertarget{generar-sitio-privado-para-documentaciuxf3n-de-estudiantes}{%
\subsection{Generar sitio privado para documentación de
estudiantes}\label{generar-sitio-privado-para-documentaciuxf3n-de-estudiantes}}

\texttt{cd\ \$(repo\ -inf)/colaborative-writing}

HedgeDoc les gustó mucho a los estudiantes del taller.

\begin{itemize}
\item
  {[}✓{]} Generar un \texttt{docker-compose.yml} con la configuración de
  \href{https://unix.stackexchange.com/questions/518849/how-does-one-set-up-codimd-as-a-personal-wiki}{CodiMD}
  o \href{https://docs.hedgedoc.org/setup/docker/}{HedgeDoc}
\item
  {[}✓{]} Configurar la instancia de \texttt{HedgeDoc} para ser privada.

  ¿Se puede autenticar?
  \href{https://docs.hedgedoc.org/guides/reverse-proxy/}{usando un proxy
  inverso}

  Se utilizó \href{https://docs.hedgedoc.org/configuration/}{la
  documentación} para
  \href{/home/x/src/gitlab.com/gcd-ibero/gestion-de-infraestructura/colaborative-writing}{configurar
  la aplicación para ser privada}.

  Se puede agregar la opción de autorizar con:
  \href{https://docs.hedgedoc.org/guides/auth/oauth/}{OAuth}
  \href{https://docs.hedgedoc.org/guides/auth/ldap-ad/}{LDAP}

  \href{https://github.com/geerlingguy/ansible-role-certbot}{Se puede
  configurar la auto-generación de certificados con Ansible}
\item
  {[}✓{]} Configurar un proxy para que la instancia de \texttt{HedgeDoc}
  funcione desde un servidor remoto.

  \begin{itemize}
  \tightlist
  \item
    Se modificaron las variables \texttt{CMD\_DOMAIN} y
    \texttt{CMD\_URL\_ADDPORT}.
  \end{itemize}
\item
  {[}✓{]} Agregar un proxy \texttt{nginx} (/ge c74c771)
\item
  {[}✗{]} Configurar \texttt{certbot} para adquirir un certificado SSL
  para \texttt{pad.haase.mx}. Usando \texttt{pad.lbustio.com} el sistema
  funcionó correctamente.

  \begin{itemize}
  \item
    {[}✓{]} El proxy de \texttt{nginx} debe
    \href{https://docs.hedgedoc.org/guides/reverse-proxy/}{generar
    ciertos encabezados para funcionar apropiadamente} (/ge 100c31a)
  \item
    {[}✗{]} Generar un contenedor \texttt{certbot} y compartir la
    configuración de certificados para que \texttt{nginx} use los
    certificados.
  \end{itemize}
\item
  {[}✓{]} Configurar nuestro proxy de nginx para redirigir el tráfico de
  \texttt{pad.lbustio.com} desde \texttt{traefik} hacia nuestro proxy.

  https://doc.traefik.io/traefik/getting-started/configuration-overview/
  https://www.digitalocean.com/community/tutorials/how-to-use-traefik-as-a-reverse-proxy-for-docker-containers-on-ubuntu-18-04

  https://www.digitalocean.com/community/tutorials/how-to-use-traefik-v2-as-a-reverse-proxy-for-docker-containers-on-ubuntu-20-04

\begin{verbatim}
$ cr /ge
$ cd escritura-colaborativa
$ docker-compose --context=lbustio up -d
\end{verbatim}

\begin{verbatim}
$ cr /ge
$ mk pad
\end{verbatim}

\begin{verbatim}
$ docker-compose --context=lbustio logs proxy
Attaching to escritura-colaborativa_proxy_1
proxy_1  | /docker-entrypoint.sh: /docker-entrypoint.d/ is not empty, will attempt to perform configuration
proxy_1  | /docker-entrypoint.sh: Looking for shell scripts in /docker-entrypoint.d/
proxy_1  | /docker-entrypoint.sh: Launching /docker-entrypoint.d/10-listen-on-ipv6-by-default.sh
proxy_1  | 10-listen-on-ipv6-by-default.sh: info: can not modify /etc/nginx/conf.d/default.conf (read-only file system?)
proxy_1  | /docker-entrypoint.sh: Launching /docker-entrypoint.d/20-envsubst-on-templates.sh
proxy_1  | /docker-entrypoint.sh: Launching /docker-entrypoint.d/30-tune-worker-processes.sh
proxy_1  | /docker-entrypoint.sh: Configuration complete; ready for start up
proxy_1  | 2021/08/13 22:26:17 [emerg] 1#1: unknown "connection_upgrade" variable
proxy_1  | nginx: [emerg] unknown "connection_upgrade" variable
\end{verbatim}

  -\textgreater{} Desafortunadamente, el proxy no está dirigiendo las
  aplicaciones como debe.

  \begin{itemize}
  \item
    🗵 Configurar \texttt{traefik} localmente.

    No será necesario jugar por el momento, porque el sistema reconoció
    \texttt{pad.lbustio.com}
  \end{itemize}
\item
  ☑ Agregar \texttt{pad.lbustio.com} al DNS.

  Se le comunicó a Lázaro que hay que actualizar el DNS.

  En \texttt{name.com} con el usuario lbustio.
\item[$\square$]
  (Opcional) Automatizar que al final de la sesión de \texttt{HedgeDoc}
  se agreguen a una página.

  Podría usarse \href{https://github.com/hedgedoc/cli}{hedgedoc/cli}
\item
  {[}?{]} Configurar el aviso para que cambien de teclado y dirección.

  \begin{itemize}
  \item
    \href{https://www.systutorials.com/docs/linux/man/1-byobu/}{Se
    pueden agregar scripts en
    \texttt{\$BYOBU\_CONFIG\_DIR/bin/\$\{WAIT\_TIME\_IN\_SECONDS\}\_\$\{SCRIPT\_NAME\}}}
    para que actualice el usuario al teclado y el usuario dirigiendo
    cada 3 minutos, con un aviso visual o auditivo muy evidente.
  \item
    {[}?{]} Hacer que inicien sesión con su nombre. Asociar su clave
    pública con el nombre en byobu.
  \item
    {[}✗{]} Entrar a la sesión restringida de \texttt{byobu}.

    La sesión no restringida de byobu nos está funcionando relativamente
    bien.
  \end{itemize}
\item
  {[}✓{]} Levantar aplicación \texttt{HedgeDoc} en
  \textless pad.lbustio.com\textgreater{}
\item
  ☑ Agregar usuarios en HedgeDoc.

  Podrían agregarse localmente los usuarios en HedgeDoc, pero también
  podría intentarse generar un SingleSignOn con el siguiente rubro,
  usando \texttt{authelia}.

  \begin{itemize}
  \item
    Podría hacerse desde la interfaz de HedgeDoc.

\begin{verbatim}
ssh lb
USER=
docker exec -ti escritura-colaborativa-app-1 bash
/hedgedoc/bin/manage_users --add ${USER}
\end{verbatim}
  \item
    Podría agregarse un proxy de identidad como
    \href{https://github.com/authelia/authelia/}{authelia}.
  \item
    Podría solicitar que usen una cuenta de GitLab y utilizar el
    servidor de autenticación de GitLab.
  \end{itemize}
\item
  ☑ Agregar autenticación para acceder a la página de datos de servidor.

  \href{https://mindup.medium.com/add-basic-authentication-in-docker-compose-files-with-traefik-34c781234970}{Una
  posible opción para resolver este problema es usar \texttt{traefik}}

  Parece que Renier intentó instalar authelia, pero no levanta.

  \$\{HOME\}/xihh/proyectos/autenticación-de-lbustio.com
\item
  ☑ Configurar \texttt{traefik} para reenviar los encabezados de la
  petición completos a hedgedoc.
\item
  ☑ Grabar la sesión del taller para que la vean si faltan.
\item
  {[}?{]}
  \href{https://docs.hedgedoc.org/guides/providing-terms/}{Podría haber
  términos de uso en el pad}.
\end{itemize}

\hypertarget{compartir-un-navegador-para-hacer-los-retos}{%
\section{Compartir un navegador para hacer los
retos}\label{compartir-un-navegador-para-hacer-los-retos}}

Se generó un contendor con un sistema operativo especializado en
ciberseguridad, que contiene las aplicacione necesarias para explorar
los juegos que se proponen y que permite compartir el control de la
máquina para trabajar en equipo.

\hypertarget{usar-servidor-para-pruxe1cticas-de-seguridad-informuxe1tica}{%
\section{Usar servidor para prácticas de seguridad
informática}\label{usar-servidor-para-pruxe1cticas-de-seguridad-informuxe1tica}}

Se generó un servidor que permite entregar la solución de los retos en
un medio competitivo.

Este servidor también permite mostrar los retos en el orden apropiado y
vincular los conocimientos con los retos.

\hypertarget{generar-sitio-de-documentaciuxf3n-para-taller-de-ciberseguridad}{%
\subsection{Generar sitio de documentación para taller de
ciberseguridad}\label{generar-sitio-de-documentaciuxf3n-para-taller-de-ciberseguridad}}

Una de las técnicas de metacognición que trabajamos durante el taller
consiste en generar un plan de trabajo para solucionar cada uno de los
juegos, donde se identifican:

\begin{itemize}
\tightlist
\item
  Objetivo
\item
  Pasos a seguir
\item
  Incógnitas
\end{itemize}

Esa herramienta se utiliza además para:

\begin{itemize}
\item
  Describir lo que se aprende en el camino, haciendo explícito el
  conocimiento y ayudando a consolidarlo.
\item
  Retomar la actividad pasado un tiempo.
\item
  Introduciendo a los integrantes nuevos del equipo.
\end{itemize}

\hypertarget{cuxf3mo-compartir-control-en-aplicaciones-gruxe1ficas}{%
\subsection{☑ Cómo compartir control en aplicaciones
gráficas}\label{cuxf3mo-compartir-control-en-aplicaciones-gruxe1ficas}}

Compartir la pantalla puede hacerse con:

\begin{itemize}
\item
  VNC server/viewer

  \begin{itemize}
  \item
    \href{https://www.realvnc.com/en/connect/download/vnc/}{Real VNC}
    (propietario)
  \item
    vinagre/vino
  \item
    tigervnc
  \item
    wayvnc
  \item
    x11vnc
  \end{itemize}
\item
  \href{http://xpra.org/}{Xpra}

  \begin{itemize}
  \tightlist
  \item
    se puede consultar directamente desde el navegador
    \href{https://github.com/Xpra-org/xpra-html5}{tiene un cliente
    HTML5}
  \item
    \href{https://github.com/Xpra-org/xpra/blob/master/README.md}{funciona
    en cualquier plataforma}.
  \end{itemize}

  Se puede iniciar desde el servidor:

\begin{verbatim}
MMAP=/run/user/1000/xpra/desktop.mmap
mkdir -p ${MMAP}
PROGRAM=dwm
xpra start-desktop --max-size=1024x768 --min-size=1024x768 --desktop-scaling=off --clipboard=yes --clipboard-direction=both --mmap=${MMAP} --start ${PROGRAM}
\end{verbatim}

  También se puede iniciar de manera remota

\begin{verbatim}
USER=cybersecurity
SERVER=192.168.90.11
PROGRAM=dwm
xpra start-desktop --max-size=1024x768 --min-size=1024x768 --desktop-scaling=off --clipboard=yes --clipboard-direction=both ssh://${USER}@${SERVER}/ --start ${PROGRAM}
\end{verbatim}

  Para usar la gui compartida:

\begin{verbatim}
MMAP=/run/user/1000/xpra/desktop.mmap
USER=
SERVER=192.168.122.120
REMOTE_DISPLAY=
xpra attach ssh://${SERVER}/${REMOTE_DISPLAY} --sharing=yes --start=firefox --clipboard=yes --clipboard-direction=both --max-size=1024x768 --min-size=1024x768 --mmap=${MMAP} # no funciona
\end{verbatim}

  Para iniciar aplicaciones:

\begin{verbatim}
APP=
USER=
SERVER=192.168.122.120
REMOTE_DISPLAY=
xpra attach --desktop-scaling=off --sharing=yes --max-size=1024x768 --min-size=1024x768 --clipboard=yes --clipboard-direction=both --start-child ${APP} ssh://${USER}@${SERVER}/${REMOTE_DISPLAY}
\end{verbatim}

  \hypertarget{problemas-con-xpra}{%
  \subsection{Problemas con xpra}\label{problemas-con-xpra}}

  Si se usan las configuraciones de ssh, la conexión falla con el error:

\begin{verbatim}
Warning: failed to connect:
 connection failed: 'canonicaldomains'
\end{verbatim}

  Los menúes de las GUIs no permanecen abiertos como en las aplicaciones
  locales
\end{itemize}

\begin{itemize}
\item[$\square$]
  Configurar un servicio en el servidor que pueda compartirse.

  \texttt{cr\ lb} (commit 4a717c)
\item
  ☑ Configurar un servicio en el servidor que pueda usar ssh desde red
  inalámbrica en ibero.

  Configurando un proxy TCP para el servidor SSH.
\item
  {[}?{]} Configurar monitoreo para redes:
\end{itemize}

\hypertarget{herramientas-para-visualizar-la-memoria-urlhttpsstackoverflow.comquestions5752605tool-to-clearly-visualize-memory-layout-of-a-c-program}{%
\subsection{\texorpdfstring{Herramientas para visualizar la memoria
(\emph{Tool to Clearly Visualize Memory Layout of a C Program},
s.~f.)}{Herramientas para visualizar la memoria (Tool to Clearly Visualize Memory Layout of a C Program, s.~f.)}}\label{herramientas-para-visualizar-la-memoria-urlhttpsstackoverflow.comquestions5752605tool-to-clearly-visualize-memory-layout-of-a-c-program}}

Dado que una de las regiones más desarrollada de nuestro cerebro es el
córtex visual, los mecanismos de visualización nos permiten comunicarnos
efectivamente y transmitir ideas complejas muy rápidamente cuando se
utilizan representaciones visuales efectivas.

Muchos de los programas que se usan en el área de ciberseguridad
muestran una representación textual de la memoria y de las instrucciones
de los programas ejecutables.

Una de las limitantes para entender qué estamos haciendo al desarrollar
estas habilidades, es que la abstracción necesaria para explorar las
aplicaciones usando herramientas con representación textual no facilita
la introducción a estas actividades.

\begin{itemize}
\item
  \href{https://www.cs.kent.ac.uk/projects/gc/gcspy/}{GCspy}
\item
  \href{http://www.cs.kent.edu/~jmaletic/cs69995-PC/papers/Moreta07.pdf}{Paper
  de visualización donde no se muestra ni se hace referencia al código}
\item
  \href{https://qira.me/}{QIRA}
\item
  \href{https://github.com/longld/peda}{PEDA}
\item
  \href{https://cutter.re/}{Cutter}

  \begin{itemize}
  \item
    \href{https://github.com/radareorg/iaito}{iaito}
  \item
    \href{https://github.com/radareorg/radare2}{radare2}
  \item
    \href{https://rizin.re/}{rizin}
  \end{itemize}
\item
  \href{https://github.com/joyent/statemap}{statemap} y
  \href{https://www.youtube.com/watch?v=x3rmg33j7RA}{el video con el que
  lo encontré}

  (parece que statemap requiere input de «instrumentation»)
\end{itemize}

\href{https://www.ubuntupit.com/best-linux-debuggers-for-modern-software-engineers/}{Varios
debuggers permiten visualizar la memoria}:

\begin{itemize}
\tightlist
\item
  \texttt{ddd}
\item
  \texttt{kgdb}
\item
  \texttt{ghidra}
\item
  \texttt{affinic-debugger}
\end{itemize}

https://stackoverflow.com/questions/5344764/gui-debugger-for-c-on-linux:

\begin{itemize}
\item
  \texttt{nemiver}
\item
  https://github.com/DataChi/memdb
\item
  https://www.xmodulo.com/visualize-memory-usage-linux.html
\end{itemize}

\href{https://github.com/eteran/edb-debugger}{\texttt{edg} está
inspirado en Ollydbg} y tiene
\href{https://github.com/eteran/edb-debugger/wiki/Stack-View}{un widget
para visualizar el stack \texttt{Stack\ View}}.

\hypertarget{dirigir-el-entrenamiento-hacia-binarios-y-cuxf3mo-funcionan-las-computadoras}{%
\subsection{Dirigir el entrenamiento hacia binarios y cómo funcionan las
computadoras}\label{dirigir-el-entrenamiento-hacia-binarios-y-cuxf3mo-funcionan-las-computadoras}}

\begin{itemize}
\item
  Ética

  \begin{itemize}
  \tightlist
  \item
    Legislación
  \item
    Responsible Disclosure
  \item
    Bug Bounties
  \end{itemize}
\item
  Ejercicios de ASM
\item
  Errores comunes en programas de C sin mitigaciones

  \begin{itemize}
  \tightlist
  \item
    Stack Overflow
  \item
    Heap Overflow
  \item
    Use after free
  \item
    Format String Vulnerability
  \end{itemize}
\item
  Ejecución con bibliotecas dinámicas

  \href{https://blog.gibson.sh/2017/11/26/creating-portable-linux-binaries/}{Un
  artículo que explica cómo usar versiones portables de las bibliotecas}

  \href{https://github.com/georgy7/build-with-musl}{Un tutorial para
  construir programas vinculados estáticamente}

  \href{https://github.com/robxu9/bash-static/blob/master/build.sh}{Un
  script para construir bash estáticamente}
\item
  Derrotar mitigaciones:

  \begin{itemize}
  \item
    ALSR

    \begin{itemize}
    \tightlist
    \item
      Hacer que el programa te diga el lugar de la memoria donde se
      cargan las funciones de la tabla global de ajustes (GOT).
    \end{itemize}
  \item
    NX

    \begin{itemize}
    \tightlist
    \item
      Escribir en la pila únicamente las direccioens del código que
      quieres ejecutar.
    \end{itemize}
  \item
    Stack Canary

    \begin{itemize}
    \item
      Hacer que el programa te diga el código secreto.

      ¿Sobreescribir cuidando mantener este código secreto?
    \end{itemize}
  \item
    \href{https://googleprojectzero.blogspot.com/2019/02/examining-pointer-authentication-on.html}{Pointer
    Autentication}

    \begin{itemize}
    \tightlist
    \item
      ¿Fuerza bruta?
    \end{itemize}
  \end{itemize}
\item
  Encontrar problemas en software

  \begin{itemize}
  \tightlist
  \item
    Fuzzers
  \end{itemize}
\item
  Return Oriented Programming

  \begin{itemize}
  \tightlist
  \item
    Escribir en la pila los lugares donde está el código que quieres
    ejecutar.
  \end{itemize}
\item
  Otros lenguajes de programación

  \begin{itemize}
  \tightlist
  \item
    eval
  \end{itemize}
\item
  HTTP session hijacking
\item
  \href{https://www.redeszone.net/tutoriales/seguridad/que-son-ataques-inyeccion-comandos/}{Inyección
  de comandos}

  \begin{itemize}
  \tightlist
  \item
    SQL injection
  \end{itemize}
\item
  Cross Site Scripting (XSS)

  \begin{itemize}
  \tightlist
  \item
    sandbox iframes
  \end{itemize}
\item
  Server Side Request Forgery
\item
  WEP/WPA cracking
\item
  \textbf{Explorar bug bounties}
\end{itemize}

\hypertarget{ejemplos-de-actividades}{%
\section{Ejemplos de actividades}\label{ejemplos-de-actividades}}

\hypertarget{desbordamiento-de-memoria-en-la-pila}{%
\subsubsection{Desbordamiento de memoria en la
pila}\label{desbordamiento-de-memoria-en-la-pila}}

Esta actividad consiste en llevar paso a paso por un ataque de
desbordamiento de memoria.

\begin{itemize}
\item
  Desarrollar un programa que por errores de programación permite
  escribir en una variable que no debería.
\item
  Desarrollar un programa permite escribir en una variable que debería
  estar inaccesible, pero que únicamente con un valor específico muestra
  la respuesta.
\item
  En el tercer programa, el valor específico es una dirección de
  memoria, que permite ejecutar una función inaccesible.
\item
  En el cuarto programa, el valor específico es una dirección de
  memoria, que se usa para el funcionamiento normal del programa.
\item
  El quinto programa, permite escribir un espacio de memoria grande,
  donde el objetivo es escribir directamente código de máquina y luego
  debe usarse la pila para dirigir la ejecución del programa a esa
  dirección de memoria.
\end{itemize}

\hypertarget{buenas-pruxe1cticas-para-contraseuxf1as}{%
\subsubsection{Buenas prácticas para
contraseñas}\label{buenas-pruxe1cticas-para-contraseuxf1as}}

Seleccionar contraseñas desde la lista de contraseñas comunes.

\begin{itemize}
\item
  Generar 3 grupos 10 contraseñas a partir de rockyou.txt:

  \begin{itemize}
  \tightlist
  \item
    4 contraseñas comunes
  \item
    2 contraseñas repetidas
  \item
    2 contraseñas con patrones

    \begin{itemize}
    \tightlist
    \item
      palabra con reemplazo
    \item
      palabra con números
    \end{itemize}
  \item
    2 contraseñas seguras

    \begin{itemize}
    \tightlist
    \item
      5 palabras aleatorias
    \item
      una cadena larga aleatoria
    \end{itemize}
  \end{itemize}
\item
  Configurar las contraseñas para usar las medidas de seguridad de cada
  uno de los niveles esperados:

  \begin{itemize}
  \item
    Cifrar (con contraseña en texto plano)

    \begin{itemize}
    \tightlist
    \item
      Generar una contraseña segura y publicarla en un archivo oculto,
    \end{itemize}
  \item
    Resumir criptográficamente (sin sal)
  \item
    Resumir criptográficamente (con sal)
  \item
    Utilizar un algoritmo difícil en memoria
  \end{itemize}
\item
  Cifrar la lista de contraseñas para el primer nivel

  \begin{itemize}
  \item
    Dejar instrucciones para descifrarlas con la contraseña oculta.

    Podrían cifrarse y descifrarse con:

    \begin{itemize}
    \item
      \href{https://askubuntu.com/questions/60712/how-do-i-quickly-encrypt-a-file-with-aes\#60713}{aescrypt}
      que \href{https://www.aescrypt.com/download/}{puede descargarse
      para cualquier SO}
    \item
      \href{https://askubuntu.com/a/60713}{openssl} que podría estar
      instalado directamente en el equipo.
    \end{itemize}
  \end{itemize}
\item
  Generar usuarios pwd1, pwd2, pwd3,

  \begin{itemize}
  \tightlist
  \item
    Configurar contraseñas para cada usuario. Este servicio permite a
    los usuarios validar su configuración de SSH y GPG.
  \end{itemize}
\end{itemize}

\hypertarget{uso-de-claves-puxfablicas}{%
\subsubsection{Uso de claves públicas}\label{uso-de-claves-puxfablicas}}

El servicio consiste en:

\begin{itemize}
\tightlist
\item
  servidor SSH con un único usuario configurado para

  \begin{itemize}
  \tightlist
  \item
    saludar
  \item
    ofrecer un mensaje cifrado con una clave pública
  \end{itemize}
\end{itemize}

La parte práctica del examen consiste en:

\begin{itemize}
\item
  Entrar al servidor con su clave de SSH.
\item
  Su mensaje está en un archivo con el hash indicado en la lista.

\begin{verbatim}
$nombre <tab> $hash
\end{verbatim}
\item
  Pueden descargar un archivo comprimido con toda la info si quieren
  trabajar localmente.
\item
  Tienen que entrar al servidor, encontrar el archivo y escribir su
  mensaje en la respuesta del examen.
\item
  Yo debo comparar el estudante con el mensaje secreto.

\begin{verbatim}
$nombre <tab> mensaje
\end{verbatim}
\end{itemize}

\hypertarget{luxednea-de-investigaciuxf3n}{%
\section{Línea de investigación}\label{luxednea-de-investigaciuxf3n}}

\hypertarget{cronograma-de-actividades}{%
\section{Cronograma de actividades}\label{cronograma-de-actividades}}

El cronograma de actividades
\href{https://docs.google.com/spreadsheets/d/1nsChpTgQXJl4jjDb0Vimj4viFeqX_fG5-iLF55dvnx0/edit?usp=sharing}{es
público en internet}.

\newpage

\hypertarget{semblanza}{%
\section{Semblanza}\label{semblanza}}

Ingeniero Biotecnólogo, especializado en bioinformática, con experiencia
gestionando infraestructura de cómputo científico y para empresas
tecnológicas.

En 2015 se encargó de la Seguridad Informática del servicio de IPTV de
Totalplay, dejando 0 vulnerabilidades conocidas e implementando un
proceso automatizado para verificar conformidad.

Fue Subdirector de Bioinformática en el Instituto Nacional de Medicina
Genómica, donde optimizó procesos de análisis para genomas completos
para aprovechar la arquitectura de cómputo científico, desarrolló una
metodología de análisis para garantizar la reproducibilidad de los
análisis; e impartió varios talleres para el uso de cómputo científico y
desarrollo de análisis bioinformáticos usando buenas prácticas de
desarrollo.

Ha desarrollado proyectos de análisis de datos en proyectos comerciales
y desarrollado análisis de redes usado algoritmos de información mutua.

Actualmente es Académico del Grupo de Ciencia de Datos de la Universidad
Iberoamericana, interesado en cómo sistematizar, automatizar y optimizar
sistemas de análisis para ayudar a resolver problemas sociales,
educativos y de salud.

\newpage

\hypertarget{referencias}{%
\section*{Referencias}\label{referencias}}
\addcontentsline{toc}{section}{Referencias}

\hypertarget{refs}{}
\begin{CSLReferences}{1}{0}
\leavevmode\hypertarget{ref-937ioE1w}{}%
(s.~f.).
\url{http://moglen.law.columbia.edu/audio/DSG-CUNY-BeforeAndAfterIP.mp3}

\leavevmode\hypertarget{ref-Z30bKsm}{}%
(s.~f.).
\url{http://publicaciones.anuies.mx/pdfs/revista/Revista62_S1A3ES.pdf}

\leavevmode\hypertarget{ref-AZg3pHkj}{}%
(s.~f.). \url{http://www.planeducativonacional.unam.mx/PDF/CAP_16.pdf}

\leavevmode\hypertarget{ref-k6AZlArO}{}%
(s.~f.). \url{https://core.ac.uk/download/pdf/33680124.pdf}

\leavevmode\hypertarget{ref-r40cJomQ}{}%
(s.~f.).
\url{https://emtemp.gcom.cloud/ngw/globalassets/en/publications/documents/top-priorities-for-it-leadership-vision-for-2021-ebook.pdf}

\leavevmode\hypertarget{ref-EObdM9yf}{}%
(s.~f.). \url{https://files.eric.ed.gov/fulltext/EJ1081990.pdf}

\leavevmode\hypertarget{ref-Rb0wL9fD}{}%
(s.~f.).
\url{https://repositorium.sdum.uminho.pt/bitstream/1822/23813/1/iceis2012_quimera_vf.pdf}

\leavevmode\hypertarget{ref-l4GIpJvw}{}%
(s.~f.).
\url{https://www.usenix.org/system/files/conference/osdi14/osdi14-paper-yuan.pdf}

\leavevmode\hypertarget{ref-w3tk3cB5}{}%
(s.~f.). \href{https://nc-broccoli}{nc-broccoli}

\leavevmode\hypertarget{ref-9rraBh7C}{}%
(s.~f.). \href{https://u-gone}{u-gone}

\leavevmode\hypertarget{ref-EYgw7dv0}{}%
Aftab, M. T., \& Tariq, M. H. (2018). Continuous Assessment as a Good
Motivational Tool in Medical Education. \emph{Acta Medica Academica},
\emph{47}(1), 76. \url{https://doi.org/10.5644/ama2006-124.216}

\leavevmode\hypertarget{ref-w5Mseb7f}{}%
\emph{ASPECTOS LEGALES RELACIONADOS CON EL DESARROLLO Y USO DEL
SOFTWARE}. (s.~f.). Recuperado 15 de agosto de 2021, de
\url{http://itcelenes.mx.tripod.com/AFIUnidad8.html}

\leavevmode\hypertarget{ref-E4qTsP6j}{}%
Bowyer, J., \& Hughes, J. (2006). Assessing undergraduate experience of
continuous integration and test-driven development. \emph{Association
for Computing Machinery (ACM)}.
\url{https://doi.org/10.1145/1134285.1134393}

\leavevmode\hypertarget{ref-isyry8r9}{}%
Bruner, J. S. (1977). \emph{The process of education}. Harvard
University Press.

\leavevmode\hypertarget{ref-mzz5Yp9s}{}%
Bruner, J. S. (1978). \emph{Toward a theory of instruction} (8. pr).
Harvard Univ. Press.

\leavevmode\hypertarget{ref-GaqreT3v}{}%
Christoforou, A. P., \& Yigit, A. S. (2008). Improving teaching and
learning in engineering education through a continuous assessment
process. \emph{European Journal of Engineering Education}, \emph{33}(1),
105-116. \url{https://doi.org/10.1080/03043790701746405}

\leavevmode\hypertarget{ref-duUamzUT}{}%
\emph{dbp.io :: Programming as Literature}. (s.~f.). Recuperado 8 de
agosto de 2021, de
\url{https://dbp.io/essays/2012-10-24-programming-literature.html}

\leavevmode\hypertarget{ref-an6jVpNC}{}%
Eddy, B. P., Wilde, N., Cooper, N. A., Mishra, B., Gamboa, V. S., Shah,
K. M., Deleon, A. M., \& Shields, N. A. (2017). A Pilot Study on
Introducing Continuous Integration and Delivery into Undergraduate
Software Engineering Courses. \emph{Institute of Electrical and
Electronics Engineers (IEEE)}.
\url{https://doi.org/10.1109/cseet.2017.18}

\leavevmode\hypertarget{ref-15VxjIHpR}{}%
Embury, S. M., \& Page, C. (2019). Effect of Continuous Integration on
Build Health in Undergraduate Team Projects. En \emph{Lecture Notes in
Computer Science}. Springer Science and Business Media LLC.
\url{https://doi.org/10.1007/978-3-030-06019-0_13}

\leavevmode\hypertarget{ref-jC7uBQxN}{}%
Fischer, K., Vaupel, S., Heller, N., Mader, S., \& Bry, F. (2021).
Effects of Competitive Coding Games on Novice Programmers. En
\emph{Advances in Intelligent Systems and Computing}. Springer Science
and Business Media LLC.
\url{https://doi.org/10.1007/978-3-030-68198-2_43}

\leavevmode\hypertarget{ref-L7gjhE6x}{}%
González Martínez, J. (2003). \emph{La producción en serie y la
producción flexible: principios, técnicas organizacionales y fundamentos
del cambio} (1. ed). Universidad Autónoma Metropolitana Azcapotzalco.

\leavevmode\hypertarget{ref-10CD5GUrI}{}%
Hernández, R. (2012). Does continuous assessment in higher education
support student learning? \emph{Higher Education}, \emph{64}(4),
489-502. \url{https://doi.org/10.1007/s10734-012-9506-7}

\leavevmode\hypertarget{ref-6qRfarph}{}%
jeronicalafell. (2017, enero 10). Frase célebre Confucio - Aprender /
Jeroni Calafell. \emph{Jeroni Calafell}.
\url{https://jeronicalafell.com/frase-celebre-confucio-aprender/}

\leavevmode\hypertarget{ref-zaYZE70v}{}%
Krafczyk, M., Shi, A., Bhaskar, A., Marinov, D., \& Stodden, V. (2019).
Scientific Tests and Continuous Integration Strategies to Enhance
Reproducibility in the Scientific Software Context. \emph{Association
for Computing Machinery (ACM)}.
\url{https://doi.org/10.1145/3322790.3330595}

\leavevmode\hypertarget{ref-61Mhr63C}{}%
Lehman, M. M. (1980). Programs, life cycles, and laws of software
evolution. \emph{Proceedings of the IEEE}, \emph{68}(9), 1060-1076.
\url{https://doi.org/10.1109/proc.1980.11805}

\leavevmode\hypertarget{ref-UmmD4xUI}{}%
Levy, F., \& Murnane, R. J. (2005). \emph{The new division of labor: how
computers are creating the next job market} (2. print. and 1. paperback
print). Russell Sage Foundation {[}u.a.{]}.

\leavevmode\hypertarget{ref-JJPwy3DY}{}%
\emph{Marco legal educativo de los Estados Unidos Mexicanos}. (s.~f.).
La concepción de la Evaluación. Recuperado 8 de agosto de 2021, de
\url{https://deliarodriguezinvestigacion.wordpress.com/category/marco-legal-educativo-de-los-estados-unidos-mexicanos/}

\leavevmode\hypertarget{ref-HDmPArpB}{}%
McConnell, S. (2004). \emph{Code complete} (2nd ed). Microsoft Press.

\leavevmode\hypertarget{ref-LX5rGLI1}{}%
México. (s.~f.). \emph{Ciberseguridad}. Recuperado 15 de agosto de 2021,
de \url{https://ciberseguridad.com/normativa/latinoamerica/mexico/}

\leavevmode\hypertarget{ref-tZcz0BiL}{}%
Piaget, J., Inhelder, B., Delval, J., \& Lomelí, P. (2016).
\emph{Psicología del niäno}.
\url{https://elibro.net/ereader/elibrodemo/116205}

\leavevmode\hypertarget{ref-BplcWPD4}{}%
Sanz-Pérez, E. S. (2019). Students' performance and perceptions on
continuous assessment. Redefining a chemical engineering subject in the
European higher education area. \emph{Education for Chemical Engineers},
\emph{28}, 13-24. \url{https://doi.org/10.1016/j.ece.2019.01.004}

\leavevmode\hypertarget{ref-9cCPbJWa}{}%
Snow, C. E. (Ed.). (1979). \emph{Talking to children: language input and
acquisition ; papers from a conference ... {[}held 6-8 September 1974
... in Boston, Massachusetts{]}} (Repr). Cambridge Univ. Press.

\leavevmode\hypertarget{ref-L4JMQzSC}{}%
Taylor, F. W., Thompson, K., \& Taylor, F. W. (2003). \emph{Scientific
management} (Repr. Taylor, New York and London: Harper Publ). Routledge.

\leavevmode\hypertarget{ref-oYJa6rHx}{}%
\emph{Tool to clearly visualize Memory Layout of a C Program}. (s.~f.).
Stack Overflow. Recuperado 30 de agosto de 2022, de
\url{https://stackoverflow.com/questions/5752605/tool-to-clearly-visualize-memory-layout-of-a-c-program}

\leavevmode\hypertarget{ref-10ZfexQUS}{}%
\emph{UAJyT}. (s.~f.). Recuperado 8 de agosto de 2021, de
\url{https://www.sep.gob.mx/es/sep1/sep1_La_Educacion_y_sus_Normas_Juridicas}

\leavevmode\hypertarget{ref-vy7gNH8L}{}%
Vygotski, L. S., Cole, M., Furio, S., John-Steiner, V., Scribner, S., \&
Souberman, E. (2009). \emph{El desarrollo de los procesos psicológicos
superiores}. Crítica.

\leavevmode\hypertarget{ref-hzq61Gnz}{}%
Zinovieva, I. S., Artemchuk, V. O., Iatsyshyn, A. V., Popov, O. O.,
Kovach, V. O., Iatsyshyn, A. V., Romanenko, Y. O., \& Radchenko, O. V.
(2021). The use of online coding platforms as additional distance tools
in programming education. \emph{Journal of Physics: Conference Series},
\emph{1840}(1), 012029.
\url{https://doi.org/10.1088/1742-6596/1840/1/012029}

\leavevmode\hypertarget{ref-11mdjMtw7}{}%
Zvacek, S., Institute for Systems and Technologies of Information,
Control and Communication, \& IEEE Education Society (Eds.). (2014).
\emph{Proceedings of the 6th International Conference on Computer
Supported Education, Barcelona, Spain, 1 - 3 April, 2014. Vol. 3: ...}
SCITEPRESS.

\end{CSLReferences}

\end{document}
